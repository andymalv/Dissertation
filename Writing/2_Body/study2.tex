\chapter{Validation of wearable sensors for use in estimating poststroke propulsion in the clinic} \label{chapter:study2}
\thispagestyle{myheadings}

% set this to the location of the figures for this chapter. it may
% also want to be ../Figures/2_Body/ or something. make sure that
% it has a trailing directory separator (i.e., '/')!
\graphicspath{{2_Body/Figures/}}

\section{Introduction}
Stroke is a leading cause of disability that results in neuromotor impairments that contribute to slower and more inefficient walking.
As a consequence, walking rehabilitation is a major focus after stroke, particularly propulsion function.
While the study and development of interventions targeting propulsion function after neurological injury is a highly active area of research \citep{Browne2017, Boutaayamou2015, Miyazaki2019, Awad2017b, Genthe2018, Kesar2009, McCain2019, Penke2019, Phadke2012, Takahashi2015}, the clinical use of these approaches is limited due to reduced access to the gold standard technology used to directly measure APGRFs \citep{Farris2015, Bowden2006, Jonkers2009, Turns2007, Bethoux2011, Franz2013a, Franz2014a, Martin2002}.
As long as the instruments used to measure propulsion remain inaccessible to clinicians, interventions targeting specific deficits in walking function will continue to be out of reach. \par

Wearable sensors are a potential solution to this issue: they have been used to collect a number of gait measurements outside the laboratory \citep{Boutaayamou2015, Miyazaki2019, Peruzzi2011, Seel2012, Yang2013}, and have proven effective at indirect measurements of ground reaction forces during walking \citep{Karatsidis2016, Ryu2018, Lim2016, Shahabpoor2018}.
Specifically, IMUs have shown to be useful in measuring propulsion metrics, such as peak propulsion magnitude and timing, and propulsion impulse, in hemiparetic walking \citep{Pieper2019,Revi2020}.
With the help of machine learning, IMUs may be the key to extending the study of paretic limb propulsion to the clinic. \par

The objective of this study is to validate the accuracy of an IMU based algorithm in estimating propulsion metrics versus the gold standard technology.
This study will also work to validate the accuracy of an IMU based algorithm in estimating the APGRF time series curve versus the gold standard technology.
In order to reduce bias towards current biomechanical understanding, no features will be chosen a priori; instead, a statistical approach will be taken to select those features that are most relevant in estimating propulsion metrics and the APGRF time series curve. \par

\section{Methods}
\subsection{Participants}

Data was collected as part of a study at the Neuromotor Recovery Laboratory at Boston University.
Seven individuals with chronic poststroke hemiparesis participated (Table \ref{table:participant_data2}).
Study inclusion criteria included: being greater than six months after stroke, ambulatory but with residual gait deficits, and having the ability to walk on a treadmill without orthotic support. Study exclusion criteria included: cerebellar stroke, lower extremity joint replacement or other orthopedic conditions that change walking ability, pain that limits walking ability, inability to communicate with investigators, neglect or hemianopia, or unexplained dizziness, and more than two falls in the previous month.
All study procedures were approved by the Institutional Review Board of Boston University.
Written informed consent was secured from all study participants prior to initiation of study procedures. \par

\begin{table}[h]
	\centering
	\resizebox{\textwidth}{!}{
		\begin{tabular}{|l|l|l|l|l|l|l|}
			\hline
			\textbf{Participant ID} & \textbf{Side of Paresis} & \textbf{Stroke Onset (y)} & \textbf{Sex} & \textbf{Age (y)} & \textbf{Height (cm)} & \textbf{Weight (kg)} \\
			\hline
			1                       & Right                    & 1                         & Male         & 62               & 181.5                & 94.5                 \\
			\hline
			2                       & Left                     & 9                         & Male         & 80               & 179.5                & 102.7                \\
			\hline
			3                       & Right                    & 8                         & Male         & 47               & 181.7                & 100.2                \\
			\hline
			4                       & Right                    & 5                         & Male         & 65               & 173.5                & 81.7                 \\
			\hline
			5                       & Left                     & 2                         & Male         & 48               & 180.3                & 111.4                \\
			\hline
			6                       & Left                     & 2                         & Male         & 47               & 177.4                & 92.3                 \\
			\hline
			7                       & Left                     & 3                         & Male         & 59               & 187.0                & 89.13                \\
			\hline
		\end{tabular}}
	\caption{Study 2 Participant Data}
	\label{table:participant_data2}
\end{table}


\subsection{Data Collection}
Kinematic data were captured by an 18-camera motion capture system (Qualisys, Göteborg, Sweden) based on the motion of reflective markers (Figure \ref{fig:marker_placement_study2}).
Single ‘landmark’ markers were placed to establish anatomy, including: the first and fifth metatarsals; the medial and lateral malleoli; the medial and lateral femoral condyles; the greater trochanters; the anterior superior iliac spines; and the iliac crests.
Multiple ‘cluster’ markers were placed to track the motion of the body segments, including: the pelvis, the upper legs (thighs), and the lower legs (shanks); two markers will also be placed on the heel to create a cluster with the metatarsal markers.
All markers were sampled at a rate of 200 Hz.

\begin{figure}[t]
	\includegraphics{Marker_Placement.png}
	\centering
	\caption{Motion Capture Marker Placement}
	\label{fig:marker_placement_study2}
\end{figure}

Participants performed 10-Meter Walk Tests, conducted by licensed physical therapists, to obtain their self-selected walking speed (i.e. comfortable walking speed, CWS).
Kinetic data was collected during two separate 2-minute bouts of treadmill walking at 80\% of their CWS, as per study protocols.
Ground reaction forces (GRFs) were collected using a dual-belt instrumented treadmill with 2 independent 6-degree of freedom force platforms (Bertec Corporation, Worthington, OH) sampling at 2000 Hz.
A total of 7 IMUs (Xsens Technologies B.V., Enschede, Netherlands), sampling at 100 Hz, were collected concurrently during the bouts of treadmill walking.
They were placed on the pelvis, as well as bilaterally on the thigh, shank and foot segments (Figure \ref{fig:imu_placement}).
The IMUs were placed so that their axes matched that of the motion capture space: medial/lateral movement was on the X axis, anterior/posterior movement on the Y axis and vertical movement on the Z axis. \par

\begin{figure}[t]
	\includegraphics{IMU_Placement.png}
	\centering
	\caption{IMU Placement}
	\label{fig:imu_placement}
\end{figure}

\subsection{Data Processing}

Marker motion capture data were cleaned in Qualysis Track Manager; marker and force plate data were filtered in Visual3D using a bi-directional Butterworth low pass filter at 30 Hz, then entered into MATLAB for further processing.
Using MATLAB 2019a (Mathworks, Natick, MA), the total number of strides per participant were combined across both walking bouts.
Each stride will be normalized to percent body weight (\%BW) as per the literature, and percent stance phase (\%SP, defined as the duration of the stance phase, from initial contact to final contact); the latter decision was made to better focus future analysis on the effects of propulsion, regardless of other factors in the gait cycle, as well as its common use by clinicians.
Propulsion metrics found to be significant as part of Study 1 were also calculated.
The IMU data were cleaned and filtered in MATLAB, then conducted into features for use in the machine learning algorithm (Table \ref{table:imu_features}).
For the first aim, the inputs to the model were the features listed in Table \ref{table:imu_features}; the response variable will be the metrics calculated in the first study (\ref{table:calculated_propulsion_metrics}).
For the second aim, the inputs to the model will be the features listed in Table \ref{table:imu_features}; the response variable will be the APGRF time series curve taken from the gold standard, i.e. the instrumented treadmill. \par


%NOTE: Cara said it was unclear that the below was for each IMU; I think the mention of the IMUs above and the figure of their placement is sufficent. Need to clarify
\begin{table}[h]
	\centering
	\begin{tabular}{|l|l|l|l|}
		\hline
		\textbf{Source} & \textbf{Definition} & \textbf{Directions} & \textbf{Features} \\
		\hline
		Accelerometer   & Linear acceleration & X, Y, Z             & 3                 \\
		\hline
		Gyroscope       & Angular velocity    & X, Y, Z             & 3                 \\
		\hline
	\end{tabular}
	\caption{IMU Features}
	\label{table:imu_features}
\end{table}

The remaining work was done in Python 3.13.2 (Python Software Foundation). \par
Feature ranking examined the importance of each predictor individually by using a LOO cross-validation method.
%TODO: paraphrase this sentence:
% This was done by evaluating the performance of the model with all the features included, then iteratively removing one feature as a time, retraining the model, and evaluating its performance on a validation set.
This approach evaluated performance of the model with all features included; features were then iteratively removed one at a time, the model was retrained and its performance was evaluated on a validation set.
This performance was measured using RMSE, the average of the squared difference between predictions and the ground truth.
The model chosen for the propulsion metrics analysis was a simple linear regression model (LinearRegression) from the popular machine learning package scikit-learn 1.6.1 \citep{Pedregosa2011}.
The three most important features were then included in the training model.
To train the propulsion metric model, a LOO cross-validation method was used.
The mean and standard deviation across all folds were reported for the estimation of the propulsion metric, as well as for the evaluation metrics. \par

The above LOO cross-validation model was also used to choose the features included in the APGRF model, replacing the simple linear regression model with a stochastic gradient descent model (SGDRegressor), also from scikit-learn.
The APGRF model was trained using the SGDRegressor model's \textit{partial\_fit} attribute; this allowed the model to be trained one stride at a time, allowing it to learn and correct from previous errors.
This was done on a training set made up of 70\% of the data; the remaining 30\% were used as part of the testing set to evaluate the model.


% Univariate feature ranking examined the importance of each predictor individually using an F-test, and then ranked features using the p-values of the F-test statistics, an alpha of 0.05 determining inclusion in the model.
% Once obtained, these features were used to train and test supervised models using linear regression and Stochastic Gradient Descent; 70\% of the data used for training and 30\% for testing to determine the generalization of the model.
% This process was completed for both the propulsion metrics, as well as for the APGRF curve. \par




\subsection{}
To determine the performance of the propulsion metrics model, root mean squared error (RMSE) and mean absolute percent error (MAPE) were used; these, as well as \(R^{2}\), were used to measure the performance of the APGRF model.
RMSE measures the average of the squared difference between the predictions and the ground truth, allowing for the detection of outliers; MAPE is the absolute value of the percent difference between the predictions and the ground truth; \(R^{2}\) uses the regression and total sum of squares to determine the variance between the model and a linear model. \par
For the first model, propulsion metrics taken from the gold standard of force plate data were used as the response variable.
For the second model, the APGRF time series curve taken from the gold standard of force plate data will be used as the response variable.
The models were then compared to those in the literature, with performance metrics as well as features used to construct the models being analyzed. \par


\section{Results}
\subsection{Propulsion Metrics Model}
% Figure for Participant 1/ME04
\begin{figure}[H]
	\includegraphics[scale=0.7]{Participant1_metrics.pdf}
	\centering
	\caption{Results of propulsion metrics model for Participant 1.}
	\centering
	\label{fig:metrics1}
\end{figure}

% Table for Participant 1/ME04
\begin{table}[H]
	\centering
	\begin{minipage}{0.45\textwidth}
		\textbf{Peak Braking Magnitude:\\}
		Foot Acc X, Thigh Gyro Z, Thigh Gyro X \\
		\begin{tabular}{m{3cm}|c|c|}
			         & Mean  & Std  \\
			\hline
			Measured & -8.17 & 0.70 \\
			Estimate & -8.18 & 0.29 \\
			MAPE     & 0.07  & 0.06 \\
			RMSE     & -0.57 & 0.47 \\
		\end{tabular}
	\end{minipage}
	\hfill
	\begin{minipage}{0.45\textwidth}
		\textbf{Braking Impulse:\\}
		Thigh Gyro Z, Foot Acc X, Foot Gyro X \\
		\begin{tabular}{m{3cm}|c|c|}
			         & Mean  & Std  \\
			\hline
			Measured & -2.76 & 0.22 \\
			Estimate & -2.75 & 0.05 \\
			MAPE     & 0.07  & 0.06 \\
			RMSE     & -0.18 & 0.15 \\
		\end{tabular}
	\end{minipage}
	\vspace{1cm}
	\begin{minipage}{0.45\textwidth}
		\textbf{Peak Propulsion Magnitude:\\}
		Thigh Acc Y, Shank Gyro X, Foot Acc X \\
		\begin{tabular}{m{3cm}|c|c|}
			         & Mean  & Std  \\
			\hline
			Measured & 4.05  & 0.70 \\
			Estimate & 4.04  & 0.26 \\
			MAPE     & 0.16  & 0.09 \\
			RMSE     & -0.63 & 0.40 \\
		\end{tabular}
	\end{minipage}
	\hfill
	\begin{minipage}{0.45\textwidth}
		\textbf{Propulsion Impulse:\\}
		Thigh Acc Y, Shank Acc Z, Shank Gyro X \\
		\begin{tabular}{m{3cm}|c|c|}
			         & Mean  & Std  \\
			\hline
			Measured & 0.70  & 0.16 \\
			Estimate & 0.69  & 0.07 \\
			MAPE     & 0.19  & 0.14 \\
			RMSE     & -0.13 & 0.10 \\
		\end{tabular}
	\end{minipage}
	\caption{Propulsion metric model results for Participant 1.}
	\label{table:metrics1}
\end{table}

Executing the feature selection for Participant 1, the three most important features to estimate peak braking magnitude were foot acceleration in the X direction, thigh gyroscopic force in the Z direction and thigh gyroscopic force in the X direction, resulting in a MAPE of 7\% and a RMSE of -0.57.
For braking impulse, they were thigh gyroscopic force in the Z direction, foot acceleration in the X direction and foot gyroscopic force in the X direction, resulting in a MAPE of 7\% and a RMSE of -0.18.
For peak propulsion magnitude, they were thigh acceleration in the Y direction, shank gyroscopic force in the X direction and foot acceleration in the X direction, resulting in a MAPE of 16\% and a RMSE of -0.63.
For propulsion impulse, they were thigh acceleration in the Y direction, shank acceleration in the Z direction and shank gyroscopic force in the X direction, resulting in a MAPE of 19\% and a RMSE of -0.13. \par

% Figure for Participant 2/ME07
\begin{figure}[H]
	\includegraphics[scale=0.7]{Participant2_metrics.pdf}
	\centering
	\caption{Results of propulsion metrics model for Participant 2.}
	\centering
	\label{fig:metrics2}
\end{figure}[H]

% Table for Participant 2/ME07
\begin{table}[H]
	\centering
	\begin{minipage}{0.45\textwidth}
		\textbf{Peak Braking Magnitude:\\}
		Thigh Gyro X, Thigh Gyro Y, Shank Gyro Z \\
		\begin{tabular}{|m{3cm}|r|r|}
			         & Mean  & Std  \\
			\hline
			Measured & -8.63 & 0.94 \\
			Estimate & -8.65 & 0.21 \\
			MAPE     & 0.09  & 0.09 \\
			RMSE     & -0.77 & 0.71 \\
		\end{tabular}
	\end{minipage}
	\hfill
	\begin{minipage}{0.45\textwidth}
		\textbf{Braking Impulse:\\}
		Thigh Gyro X, Thigh Gyro Y, Shank Gyro Z \\
		\begin{tabular}{m{3cm}|c|c|}
			         & Mean  & Std  \\
			\hline
			Measured & -2.57 & 0.40 \\
			Estimate & -2.58 & 0.12 \\
			MAPE     & 0.14  & 0.14 \\
			RMSE     & -0.33 & 0.30 \\
		\end{tabular}
	\end{minipage}
	\vspace{1cm}
	\begin{minipage}{0.45\textwidth}
		\textbf{Peak Propulsion Magnitude:\\}
		Shank Acc Y, Foot Acc X, Thigh Acc Y \\
		\begin{tabular}{m{3cm}|c|c|}
			         & Mean  & Std  \\
			\hline
			Measured & 6.07  & 0.97 \\
			Estimate & 6.07  & 0.36 \\
			MAPE     & 0.13  & 0.10 \\
			RMSE     & -0.79 & 0.59 \\
		\end{tabular}
	\end{minipage}
	\hfill
	\begin{minipage}{0.45\textwidth}
		\textbf{Propulsion Impulse:\\}
		Shank Acc Y, Shank Gyro Z, Foot Gyro Y \\
		\begin{tabular}{m{3cm}|c|c|}
			         & Mean  & Std  \\
			\hline
			Measured & 1.88  & 0.37 \\
			Estimate & 1.88  & 0.18 \\
			MAPE     & 0.16  & 0.13 \\
			RMSE     & -0.28 & 0.21 \\
		\end{tabular}
	\end{minipage}
	\caption{Propulsion metric model results for Participant 2.}
	\label{table:metrics2}
\end{table}

Executing the feature selection for Participant 2, the three most important features to estimate peak braking magnitude were thigh gyroscopic force in the X direction, thigh gyroscopic force in the Y direction and shank gyroscopic force in the Z direction, resulting in a MAPE of 9\% and a RMSE of -0.77.
For braking impulse, they were thigh gyroscopic force in the Y direction, thigh gyroscopic force in the X direction and shank gyroscopic force in the Z direction, resulting in a MAPE of 14\% and a RMSE of -0.33.
For peak propulsion magnitude, they were shank acceleration in the Y direction, foot acceleration in the X direction and thigh acceleration in the Y direction, resulting in a MAPE of 13\% and a RMSE of -0.79.
For propulsion impulse, they were shank acceleration in the Y direction, shank gyroscopic force in the Z direction and foot gyroscopic force in the Y direction, resulting in a MAPE of 16\% and a RMSE of -0.28. \par

% Figure for Participant 3/ME14
\begin{figure}[H]
	\includegraphics[scale=0.7]{Participant3_metrics.pdf}
	\centering
	\caption{Results of propulsion metrics model for Participant 3.}
	\centering
	\label{fig:metrics3}
\end{figure}

% Table for Participant 3/ME14
\begin{table}[H]
	\centering
	\begin{minipage}{0.45\textwidth}
		\textbf{Peak Braking Magnitude:\\}
		Shank Acc Y, Shank Acc X, Thigh Acc Y \\
		\begin{tabular}{|m{3cm}|r|r|}
			         & Mean   & Std  \\
			\hline
			Measured & -15.19 & 1.35 \\
			Estimate & -15.19 & 0.49 \\
			MAPE     & 0.07   & 0.06 \\
			RMSE     & -1.06  & 0.81 \\
		\end{tabular}
	\end{minipage}
	\hfill
	\begin{minipage}{0.45\textwidth}
		\textbf{Braking Impulse:\\}
		Shank Acc Y, Shank Gyro X, Foot Acc Y \\
		\begin{tabular}{m{3cm}|c|c|}
			         & Mean  & Std  \\
			\hline
			Measured & -3.96 & 0.37 \\
			Estimate & -3.96 & 0.09 \\
			MAPE     & 0.08  & 0.07 \\
			RMSE     & -0.29 & 0.24 \\
		\end{tabular}
	\end{minipage}
	\vspace{1cm}
	\begin{minipage}{0.45\textwidth}
		\textbf{Peak Propulsion Magnitude:\\}
		Thigh Gyro X, Shank Gyro Y, Foot Gyro Y \\
		\begin{tabular}{m{3cm}|c|c|}
			         & Mean  & Std  \\
			\hline
			Measured & 13.67 & 1.04 \\
			Estimate & 13.66 & 0.19 \\
			MAPE     & 0.06  & 0.05 \\
			RMSE     & -0.85 & 0.65 \\
		\end{tabular}
	\end{minipage}
	\hfill
	\begin{minipage}{0.45\textwidth}
		\textbf{Propulsion Impulse:\\}
		Shank Gyro Y, Foot Gyro Y, Thigh Gyro Y \\
		\begin{tabular}{m{3cm}|c|c|}
			         & Mean  & Std  \\
			\hline
			Measured & 3.43  & 0.25 \\
			Estimate & 3.43  & 0.06 \\
			MAPE     & 0.06  & 0.06 \\
			RMSE     & -0.19 & 0.17 \\
		\end{tabular}
	\end{minipage}
	\caption{Propulsion metric model results for Participant 3.}
	\label{table:metrics3}
\end{table}

Executing the feature selection for Participant 3, the three most important features to estimate peak braking magnitude were shank acceleration in the Y direction, shank acceleration in the X direction and thigh acceleration in the Y direction, resulting in a MAPE of 7\% and a RMSE of -1.06.
For braking impulse, they were shank acceleration in the Y direction, shank gyroscopic force in the X direction and foot acceleration in the Y direction, resulting in a MAPE of 8\% and a RMSE of -0.29.
For peak propulsion magnitude, they were thigh gyroscopic force in the X direction, shank gyroscopic force in the Y direction and foot gyroscopic force in the Y direction, resulting in a MAPE of 6\% and a RMSE of -0.85.
For propulsion impulse, they were shank gyroscopic force in the Y direction, foot gyroscopic force in the Y direction and thigh gyroscopic force in the Y direction, resulting in a MAPE of 6\% and a RMSE of -0.19. \par


% Figure for Participant 4/ME15
\begin{figure}[H]
	\includegraphics[scale=0.7]{Participant4_metrics.pdf}
	\centering
	\caption{Results of propulsion metrics model for Participant 4.}
	\centering
	\label{fig:metrics4}
\end{figure}

% Table for Participant 4/ME15
\begin{table}[H]
	\centering
	\begin{minipage}{0.45\textwidth}
		\textbf{Peak Braking Magnitude:\\}
		Thigh Gyro X, Shank Gyro Z, Foot Gyro Y \\
		\begin{tabular}{|m{3cm}|c|c|}
			         & Mean   & Std  \\
			\hline
			Measured & -11.29 & 1.61 \\
			Estimate & -11.27 & 0.29 \\
			MAPE     & 0.13   & 0.12 \\
			RMSE     & -1.39  & 1.03 \\
		\end{tabular}
	\end{minipage}
	\hfill
	\begin{minipage}{0.45\textwidth}
		\textbf{Braking Impulse:\\}
		Thigh Gyro Y, Thigh Acc Z, Shank Acc Z \\
		\begin{tabular}{m{3cm}|c|c|}
			         & Mean  & Std  \\
			\hline
			Measured & -3.22 & 0.45 \\
			Estimate & -3.22 & 0.15 \\
			MAPE     & 0.12  & 0.13 \\
			RMSE     & -0.37 & 0.30 \\
		\end{tabular}
	\end{minipage}
	\vspace{1cm}
	\begin{minipage}{0.45\textwidth}
		\textbf{Peak Propulsion Magnitude:\\}
		Thigh Gyro Y, Shank Acc X, Shank Gyro Z \\
		\begin{tabular}{m{3cm}|c|c|}
			         & Mean  & Std   \\
			\hline
			Measured & 5.63  & 1.17  \\
			Estimate & 6.29  & 10.39 \\
			MAPE     & 0.20  & 0.17  \\
			RMSE     & -1.05 & 0.73  \\
		\end{tabular}
	\end{minipage}
	\hfill
	\begin{minipage}{0.45\textwidth}
		\textbf{Propulsion Impulse:\\}
		Thigh Gyro Y, Shank Acc X, Shank Gyro Y \\
		\begin{tabular}{m{3cm}|c|c|}
			         & Mean  & Std  \\
			\hline
			Measured & 1.37  & 0.47 \\
			Estimate & 1.38  & 0.18 \\
			MAPE     & 0.36  & 0.31 \\
			RMSE     & -0.42 & 0.29 \\
		\end{tabular}
	\end{minipage}
	\caption{Propulsion metric model results for Participant 4.}
	\label{table:metrics4}
\end{table}

Executing the feature selection for Participant 4, the three most important features to estimate peak braking magnitude were thigh gyroscopic force in the X direction, shank gyroscopic force in the Z direction and foot gyroscopic force in the Y direction, resulting in a MAPE of 13\% and a RMSE of -1.39.
For braking impulse, they were thigh gyroscopic force in the Y direction, thigh acceleration in the Z direction and shank acceleration in the Z direction, resulting in a MAPE of 12\% and a RMSE of -0.37.
For peak propulsion magnitude, they were thigh gyroscopic force in the Y direction, shank acceleration in the X direction and shank gyroscopic force in the Z direction, resulting in a MAPE of 20\% and a RMSE of -1.05.
For propulsion magnitude, they were thigh gyroscopic force in the Y direction, shank acceleration in the X direction and shank gyroscopic force in the Y direction, resulting in a MAPE of 36\% and a RMSE of -0.42. \par


% Figure for All Participants
\begin{figure}[H]
	\includegraphics[scale=0.7]{All_Participants_metrics.pdf}
	\centering
	\caption{Results of propulsion metrics model for all participants.}
	\centering
	\label{fig:metrics_all}
\end{figure}

% Table for All Participants
\begin{table}[H]
	\centering
	\begin{minipage}{0.45\textwidth}
		\textbf{Peak Braking Magnitude:\\}
		Shank Gyro Y, Thigh Acc X, Foot Gyro Y \\
		\begin{tabular}{|m{3cm}|r|r|}
			         & Mean   & Std  \\
			\hline
			Measured & -11.22 & 2.76 \\
			Estimate & -11.22 & 0.33 \\
			MAPE     & 0.22   & 0.16 \\
			RMSE     & -2.33  & 1.49 \\
		\end{tabular}
	\end{minipage}
	\hfill
	\begin{minipage}{0.45\textwidth}
		\textbf{Braking Impulse:\\}
		Thigh Gyro X, Foot Acc Y, Thigh Acc X \\
		\begin{tabular}{m{3cm}|c|c|}
			         & Mean  & Std  \\
			\hline
			Measured & -2.97 & 0.65 \\
			Estimate & -2.97 & 0.39 \\
			MAPE     & 0.15  & 0.12 \\
			RMSE     & -0.43 & 0.30 \\
		\end{tabular}
	\end{minipage}
	\vspace{1cm}
	\begin{minipage}{0.45\textwidth}
		\textbf{Peak Propulsion Magnitude:\\}
		Thigh Acc Y, Foot Gyro X, Shank Acc Y \\
		\begin{tabular}{m{3cm}|c|c|}
			         & Mean  & Std  \\
			\hline
			Measured & 8.42  & 3.71 \\
			Estimate & 8.43  & 1.77 \\
			MAPE     & 0.43  & 0.37 \\
			RMSE     & -2.90 & 1.71 \\
		\end{tabular}
	\end{minipage}
	\hfill
	\begin{minipage}{0.45\textwidth}
		\textbf{Propulsion Impulse:\\}
		Thigh Acc Y, Foot Gyro X, Foot Acc Y \\
		\begin{tabular}{m{3cm}|c|c|}
			         & Mean  & Std  \\
			\hline
			Measured & 2.17  & 1.02 \\
			Estimate & 2.17  & 0.59 \\
			MAPE     & 0.49  & 0.48 \\
			RMSE     & -0.74 & 0.42 \\
		\end{tabular}
	\end{minipage}
	\caption{Propulsion metric model results for all participants.}
	\label{table:metrics4}
\end{table}

Executing the feature selection for the combination of all participants, the three most important features to estimate peak braking magnitude were shank gyroscopic force in the Y direction, thigh acceleration in the X direction and foot gyroscopic force in the Y direction, resulting in a MAPE of 22\% and a RMSE of -2.33.
For braking impulse, they were thigh gyroscopic force in the X direction, foot acceleration in the Y direction and thigh acceleration in the X direction, resulting in a MAPE of 15\% and a RMSE of -0.43.
For peak propulsion magnitude, they were thigh acceleration in the Y direction, foot gyroscopic force in the X direction and shank acceleration in the Y direction, resulting in a MAPE of 43\% and a RMSE of -2.90.
For propulsion impulse, they were thigh acceleration in the Y direction, foot gyroscopic force in the X direction and foot acceleration in the Y direction, resulting in a MAPE of 49\% and a RMSE of -0.74. \par

%TODO: get metrics for apgrf models
\subsection{APGRF Model}
% Figure for Participant 1/ME04
\begin{figure}[H]
	\includegraphics[width=\maxwidth{6in}]{ME04_apgrf.png}
	\centering
	% \begin{tabular}{l|l|l}
	% 	\textbf{MAPE:} 1.30 & \textbf{RMSE:} 2.54 & \textbf{\(R^{2}\):} 0.50 \\
	% \end{tabular}
	\caption{Results of APGRF model for Participant 1.}
	\centering
	\label{fig:apgrf1}
\end{figure}

When including all features in the estimation of the APGRF for Participant 1, the resulting MAPE was 130\%, the RMSE as 2.54 and the amount of variance explained was 50\%.

% Figure for Particiapnt 2/ME07
\begin{figure}[H]
	\includegraphics[width=\maxwidth{6in}]{ME07_apgrf.png}
	\centering
	% \begin{tabular}{l|r|l|r|l|r}
	% 	MAPE & 1.21 & RMSE & 1.87 & \(R^{2}\) & 0.85 \\
	% \end{tabular}
	\caption{Results of APGRF model for Participant 2.}
	\centering
	\label{fig:apgrf2}
\end{figure}

When including all features in the estimation of the APGRF for Participant 2, the resulting MAPE was 121\%, the RMSE was 1.87 and the amount of variance explained was 85\%. \par

% Figure for Participant 3/ME14
\begin{figure}[H]
	\includegraphics[width=\maxwidth{6in}]{ME14_apgrf.png}
	\centering
	% \begin{tabular}{l|r|l|r|l|r}
	% MAPE & 1.07 & RMSE & 2.44 & \(R^{2}\) & 0.92 \\
	% \end{tabular}
	\caption{Results of APGRF model for Participant 3.}
	\centering
	\label{fig:apgrf3}
\end{figure}


When including all features in the estimation of the APGRF for Participant 3, the resulting MAPE was 107\%, the RMSE was 2.44 and the amount of variance explained was 92\%. \par

% Figure for Participant 4/ME15
\begin{figure}[H]
	\includegraphics[width=\maxwidth{6in}]{ME15_apgrf.png}
	\centering
	% \begin{tabular}{l|r|l|r|l|r}
	% 	MAPE & 2.69 & RMSE & 2.81 & \(R^{2}\) & 0.71 \\
	% \end{tabular}
	\caption{Results of APGRF model for Participant 4.}
	\centering
	\label{fig:apgrf4}
\end{figure}

When including all features in the estimation of the APGRF for Participant 4, the resulting MAPE was 269\%, the RMSE was 2.81, and the amount of variance explained was 71\%. \par

% Figure for All Participants
\begin{figure}[H]
	\includegraphics[width=\maxwidth{6in}]{ME_All_apgrf.png}
	\centering
	% \begin{tabular}{l|r|l|r|l|r}
	% 	MAPE & 0.84 & RMSE & 2.94 & \(R^{2}\) & 0.88 \\
	% \end{tabular}
	\caption{Results of APGRF model for all participants.}
	\centering
	\label{fig:apgrf_all}
\end{figure}

When including all features in the estimation of the APGRF for the combined set of all participants, the resulting MAPE was 84\%, the RMSE was 2.94 and the amount of variance explained was 88\%. \par

\begin{table}[H]
	\centering
	\begin{tabular}{l|r|r|r}
		\textbf{Participant} & \textbf{MAPE} & \textbf{RMSE} & \textbf{\(R^{2}\)} \\
		\hline
		1                    & 1.30          & 2.54          & 0.50               \\
		\hline
		2                    & 1.21          & 1.87          & 0.85               \\
		\hline
		3                    & 1.07          & 2.44          & 0.92               \\
		\hline
		4                    & 2.69          & 2.81          & 0.71               \\
		\hline
		All                  & 0.84          & 2.94          & 0.88               \\
	\end{tabular}
	\caption{Evaluation metrics for results of APGRF models.}
	\label{table:apgrf_metrics}
	\centering
\end{table}

\section{Discussion}
\subsection{Propulsion Metrics Model}
The goal of this study was to allow statistical methods, particularly LOO feature selection and linear regression, to select the most significant features from wearable sensors that were most significant in estimating APGRF curve metrics.
This would allow human bias to be removed as much as possible, while allowing comparison to the current literature as to the best features produced by wearable sensors to estimate APGRF metrics. \par

The results of this study show that the significant features from wearables sensors differ at both the metric and participant level; there were few, if any, consistent selections to estimate a particular metric across all participants or to explain multiple metrics within the same participant.
In addition, the MAPE and RMSE for each model fell outside the stated goal of \textless 5\%. \par

Interestingly, when examining the model trained on all participants, thigh acceleration in the Y (anterior/posterior) direction and foot gyroscopic force in the X (medial/lateral) direction were both selected to estimate peak propulsion magnitude and propulsion impulse; this follows previous literature identifying limb acceleration and ankle moment as key contributors to walking biomechanics \citep{Zelik2016,Hsiao2016a}.
However, these produced the largest errors when compared to the measured values. \par

While this study had negative results, there remain potential avenues for further exploration.
First, a larger dataset would likely be beneficial as statistical and machine learning methods are better suited for larger amounts of data.
Second, it is possible that using a LOO method for both the feature selection and the model training resulted in a model that was overfitting, leaving it susceptible to outliers; as poststroke walking often leads to more variance in APGRFs, this also would affect the model's ability to perform.
The variability of poststroke walking compensatory strategies also introduces complexities in training statistical and machine learning models; perhaps classifying participants into groups based on these strategies or other similarities may lead to more accurate models.
Finally, while allowing for the most prominent features to be selected via statistical and machine learning methods can reduce human bias, the features were limited by the wearable sensors technology; this includes drift in the component signals and the need for them to be securely mounted to the body segments to minimize soft tissue artifacts and avoid sensors moving relative to the body segment. \par

\subsection{APGRF Models}
The goal of this study was to allow statistical and machine learning methods to select the most significant features from wearable sensors that most accurately estimate APGRF curves.
This would allow human bias to be removed as much as possible, while allowing comparison to the current literature as to the optimal wearable sensor features to estimate APGRFs. \par

The results of this study show that this approach is entirely insufficient.
Even when including all available features, the APGRF estimations were not serviceable.
This may be due in part to APGRFs being time-series data, i.e. each point follows the previous in time and that the order of the data is vital.
A large majority of statistical and machine learning approaches are not suitable for time-series data; those that are often employ auto-regression and moving averages to help estimate future data.
These methods were not used in this study as the stated goal was to estimate APGRF curves using only data from wearable sensors.
This was done in order to identify the most relevant data from wearable sensors in estimating poststroke walking biomechanics. \par

While this study had negative results, there remain potential avenues for further exploration.
First, a larger dataset would likely be beneficial as statistical and machine learning methods are better suited for larger amounts of data.
Second, more work can be done to explore potential solutions to estimating time-series data without the use of past data; more advanced techniques such as NNs or large, pretrained models may be solutions. \par
