\chapter{Conclusions}
\label{chapter:Conclusions}
\thispagestyle{myheadings}

% set this to the location of the figures for this chapter. it may
% also want to be ../Figures/2_Body/ or something. make sure that
% it has a trailing directory separator (i.e., '/')!
\graphicspath{{4_Conclusion/Figures/}}

The combined goals of these studies were to help validate the existing literature as to the most important factors in long distance walking function, as well as validate technologies to help clinicians measure these factors in the clinic.
Towards the first goal, the first study was informative: while finding that propulsion metrics in the paretic limb do contribute, braking metrics in both the paretic and nonparetic limbs may be more important factors.
While it remains important to address propulsion deficits, this study illustrates that improving braking function must also be considered when determining long distance walking function poststroke. \par

With regards to the second goal, the second study provided no positive results.
The first aim of estimating propulsion metrics using IMU data was accurate, but did not achieve the stated goal of being below a 5\% error rate; while a slightly higher error rate, such as those of some participant specific metrics, may be useful in the clinic, most were much higher.
The second aim of estimating APGRF time series curves using IMU data was entirely inaccurate; attempting to reduce human bias proved to not be nearly as useful as having domain specific knowledge to craft a biomechanically relevant approach \citep{Revi2020} \par

These studies show that, much like negative results, human bias has its use in science.

