\chapter{Introduction}
\label{chapter:introduction}
\thispagestyle{myheadings}

\section{Human Walking Biomechanics}
% % intro to the importance of walking in general 
% main points: we've been doing it forever; economic efficeny vs chimps (evolution); important to current day adults; important to health
Walking is an important function in human daily living.
Stretching back millions of years, humans have engaged in bipedal locomotion as a primary form of moving between two points \citep{Zollikofer2005}.
Through evolution, humans have prioritized walking, shown by the significant increase in bipedal walking economy compared to their evolutionary cousins, chimpanzees \citep{Sockol2007}.
This importance has stretched to the present day: a large and increasing portion of adults in the United States walk either for leisure or transportation \citep{Kruger2008, Berrigan2012}. 
Walking has an important impact on public health as an easily accessible form of moderate exercise than can reduce rates of chronic disease and medical expenditures \citep{Lee2008}. \par

% % intro to general walking biomechanics	
% mention six determinants of gait
The efficiency of human walking gait is integral to its importance, and there have been multiple theories that look to explain its mechanics \citep{Perry1992}.
\citet{Saunders1953} introduced the "six determinants of gait," wherein the goals of locomotion are to minimize the metabolic energy expenditure and the displacement of the body's center of mass (CoM) from a straight line of progression.
For a time, this was accepted as true; however, more recent studies have shown that there is little evidence that reducing CoM displacement is important to human walking \citep{DellaCroce2001,Kerrigan2001}.
In fact, attempting to reduce CoM displacement leads to increased energy expenditure \citep{Ortega2005,Massaad2010, Gordon2009}. \par

% talk about inverted pendulum model
Another theory that attempts to explain the efficiency of walking using an inverted pendulum model. 
As explained by \citep{Kuo2010}, the stance limb acts as an inverted pendulum, compensating changes in kinetic energy with changes in gravitational energy, minimizing the mechanical work done by the muscles.
This is achieved in part by keeping the knee of the stance limb extended to reduce the moment force about the knee, reducing the amount of muscular force needed to support the body weight.
Similar applications of conservation of energy are found in the swing limb, which follows a non-inverted pendulum like motion.
The principal issue with this model, however, is that it does not explain why increasing walking speed increases energy expenditure; in a true pendulum, there is no force required. 
While this model helps explain how walking can be economical, it fails to describe why there is any energetic cost to walking. \par

% intro to dynamic walking model: McGreer passive walking machine
Dynamic walking - locomotion generated primarily thought the passive dynamics of the legs, regardless of the application of power or control - takes from the inverted pendulum model to explain how the gait cycle can occur through passive dynamics through the study of passive walking machines \citep{Kuo2010, McGreer1990}.
In this model, the single limb support phase is similar to that of the inverted pendulum model, in that it requires the CoM velocity transitions from one inverted pendulum arc to the next.
This occurs because the CoM velocity is directed forward and downward due to being somewhat perpendicular to the trailing limb (the previous stance leg). 
The new arc begins with a forward and upward velocity, following the leading limb (the next stance leg) \citep{Kuo2010}. 
The dynamic walking model explains this redirection as a collision between the leading limb and the ground, dissipating energy; the resulting ground reaction force acts, in part, against the CoM velocity, requiring positive work to compensate \citep{McGreer1990}.
For passive walking machines, this is supplied by the gravitational potential energy of walking down a ramp \citep{McGreer1990}. 
When walking on level surfaces, dynamic walking robots have shown this energy can come from ankle pushoff or the hip \citep{Kuo2005, Collins2005, Kuo2002}. \par 

% dynamic walking as human walking
The theory of dynamic walking can be applied to human walking as well \citep{Kuo2010}.
This is explained by \citep{Kuo2005} through four events: collision, rebound, preload and pushoff. 
Collision, or initial contact, is defined as when the heel of the leading limb strikes the ground and performs negative work on the CoM with the ankle and knee joints. 
Quadriceps activation and knee extension of the leading limb mark rebound, which can be aided by the hips performing positive work accelerating the opposite leg through the swing phase. 
Preload occurs after the midpoint of the stance phase where the ankle joint performs negative CoM work.
The Achilles tendon likely aids in elastic energy storage from collision, allowing for an increased pushoff duration over both rebound and preload; for the inverted pendulum motion to contribute energy to ankle muscles for pushoff; and assisting in slowing the inverted pendulum motion resulting in reduced CoM velocity and less energy loss during collision. 
The cycle of reduction and restoration of energy requires active muscles with associated metabolic costs; this is called the step-to-step transition cost of human walking  \citep{Kuo2010}.
A large part of gait is focused on reducing the energy cost of step-to-step transitions \citep{Donelan2002a}. \par 

% introduce the importance of propulsion
% this sentence is taken from the PA paper; paraphrase more
During this step-to-step transition of each gait cycle, a braking force is generated by the leading limb as it makes contact with the ground in front of the body during the rebound event.
In order to move the CoM into the next step, a pushoff force produced by the trailing limb is required \citep{Donelan2002a, Kuo2010, Zelik2016}.
% a lot of this takes from Zelik and Adamcyzk; need to make sure it's paraphrased enough
These pushoff, or propulsion forces have shown to be important as they are the main sources of positive power during step-to-step transitions  \citep{Cappozzo1976, Winter1983,Hof1992}.
This burst of power comes largely from the plantarflexor muscles (the soleus, medial and lateral gastrocnemius) and tendons that work about the ankle joint; the hip muscles (illiopsoas and others) contribute less so. 
Coordination of the timing and magnitude play a vital role in step-to-step transitions. 
Through increases in speed, propulsion magnitude and onset are adjusted; propulsion magnitude increases, while propulsion onset - often occurring just before contralateral heel strike - begins earlier \citep{Kuhman2019}.
Mistimed and improper propulsion may contribute to inefficiencies and increased metabolic costs \citep{Kuo2010, Mian2006, Kramer2016}.
The knee and foot perform net negative work during this phase of the gait cycle \citep{Zelik2015, Zelik2016}. \par 



% the following is taken from the unified perspective on ankle pushoff
As shown, propulsion is a key factor in human walking. 
This is in part due to its effect on both the individual limb and the body's CoM \citep{Zelik2016}. 
The first of these is explained by the large majority of energy produced by the ankle plantarflexors; it is stored in the trailing limb as it enters swing phase, with a minimal amount moving across the hip joint to the head, arms and trunk \citep{Winter1978}. 
However, it has been shown that the trailing limb works on the CoM during the step-to-step transition, increasing its speed and kinetic energy \citep{Donelan2002}.
Both of these occurrences can be explained when it is considered that, while the limb is a small portion of the total body mass, it contributes substantially to the dynamics of the body through its localized acceleration \citep{Zelik2016}. \par

Considering the body as a system of segments, \citep{Zelik2016} proposed an equation (Equation \ref{eq_unified_acc}) that helps to relay this point. In it, the body is divided into two segments: the pushoff limb  (mass \(m_{limb}\)) and the remainder of the body (mass \(m_{ROB}\)). The position of the full body CoM  (mass \(M = m_{limb} + m_{ROB}\), position $\vec{r}_{ROB}$) is defined by the position of the CoM of the two segment groups: \par

% Equation 1, zelikUnifiedPerspectiveAnkle2016
\begin{equation}\label{eq_unified_pos} % set label for equation
	\vec{r}_{CoM} = \vec{r}_{limb}(m_{limb}/M) + \vec{r}_{ROB}(m_{ROB}/M)
\end{equation}

Taking consecutive time derivatives of CoM position produces CoM acceleration: \par

% Equation 2, zelikUnifiedPerspectiveAnkle2016
\begin{equation}\label{eq_unified_acc} % set label for equation
	\vec{a}_{CoM} = \vec{a}_{limb}(m_{limb}/M) + \vec{a}_{ROB}(m_{ROB}/M)
\end{equation}


As shown in Equation \ref{eq_unified_acc}, the acceleration of the CoM ($\vec{a}_{CoM}$) is affected by both the accelerations of the trailing limb and the remainder of the body, in relation to their contribution to the total body mass ($\vec{a}_{limb}(m_{limb}/M) + \vec{a}_{ROB}(m_{ROB}/M)$). 
As the trailing limb accounts for relatively little of the total body mass, it may be determined that the remainder of the body would be the primary contributor to $\vec{a}_{CoM}$; however, this would not be the case if $\vec{a}_{limb}$ is sufficiently large when compared to $\vec{a}_{ROB}$. 
This so happens to be true in the case of human walking: $\vec{a}_{limb}$ is large, directed forward and upward; $\vec{a}_{ROB}$ is small and works in the opposing direction \citep{Lipfert2014}. 
This equation shows that, with a large enough change in velocity or position, the smaller mass of the trailing limb can be a vital contributor to whole-body energy change. 
This allows propulsion to be a major factor in the flow of the gait cycle, creating walking patterns that are fast, stable and efficient \citep{Browne2017, Kuo2010}. \par


% % transition b/w healthy and stroke walking [taken from PA, Motor Control papers and Lou AHA grant]
However, the inability to efficiently transition from step to step is commonly seen in populations with neuromotor walking deficits - including stroke \citep{Bethoux2011, Mahon2015, Hass2005}. 
Stroke is one of the leading causes of long-term disability in the United States. It is estimated that 7 million Americans have had a stroke, with about 795,000 new or recurrent occurrences each year; it is projected that an additional 3.4 million people will have a stroke by 2030 \citep{Virani2020}. % check citation of AHA 
A majority of poststroke individuals have a walking gait that is labor intensive and inefficient; walk at speeds that are not safe for walking in the community; and have much reduced physical activity, leading to severe effects on their physical healthy and quality of life \citep{Reisman2009, Farris2015, Awad2016, Duncan2011, English2014, Michael2005, Rand2010}.
While the primary goal of stroke rehabilitation is the return to efficient walking patterns, current therapies are insufficient in reducing this disability \citep{Dickstein2008, Danks2014, Dean2012, Sullivan2014}. \par

% affects of stroke on propulsion 
A common characteristic of post-stroke walking is the inability of the paretic limb to provide pushoff work, due in part to impaired propulsion function resulting from aging or neurological injury \citep{Chen2005, Jonkers2009, Bowden2006, Franz2016}.
As ankle plantarflexors are the primary generators of propulsive power during walking, impaired plantarflexion function can lead to inefficient walking patterns.
This impairment results in the redistribution of positive work from the ankle joint to the knee and hip joints, due to mistimed and insufficient propulsion; this distal-to-proximal redistribution of positive work does not fully offset the reduction in plantarflexor work \citep{Bowden2006, Chen2008, Farris2015}. 
This leads to a lack of kinetic energy in the paretic limb at toe-off (the beginning of swing phase) reducing knee flexion both at toe-off and during swing, resulting in increased functional leg length during swing \citep{Chen2005, Balaban2014}. 
To compensate, subjects often raise their trunk during pre-swing and swing - in what is referred to as "hip hiking" - and increase lateral movement of the foot during swing - known as circumduction. 
Both of these strategies help to provide floor clearance of the limb to avoid tripping \citep{Chen2005, Balaban2014}. 
During the stance phase, the impaired single limb support of the paretic limb leads to an increase in the non-paretic pre-swing kinetic energy and decrease in swing time due to weakness or poor balance; this may also be compensated for with larger step widths \citep{Chen2005}. \par

Trailing limb angle also plays an important role in propulsion function, translating ankle plantarflexion torque into propulsion.
Healthy individuals modulate both trailing limb angle (TLA) and ankle moment to increase propulsive force; individuals post-stroke rely more heavily on changes in TLA to modulate speed, perhaps due to impaired plantarflexor strength \citep{Hsiao2015a, Hsiao2015}.
This leads to compensatory reliance on the non-paretic limb for propulsion that is characteristic of post-stroke hemiparesis; paretic limb propulsion can be up to 68\% less than that of the non-paretic limb, and is linked to impaired walking function and hemiparetic severity \citep{Awad2015, Bowden2008, Hsiao2016a, Bowden2006, Farris2015, Jonkers2009,Turns2007}.
Within the post-stroke population, levels of paretic propulsion are influential: those who are considered unlimited community ambulators based on their walking speed and distance have relatively high levels of paretic propulsion, while the paretic propulsion of home ambulators is substantially lower \citep{Bowden2013, Hsiao2016a, Hsiao2016, Fulk2017}.
Deficits in paretic propulsion have also been related to long distance walking function - a crucial determinant of community participation and perceived quality of life \citep{Awad2015a,Browne2017,Neptune2001, Combs2013}.  \par

% differences b/w healthy and stroke walking (w/ no emphasis on propulsion) [taken from Motor Control]
Differences in kinematic, kinetic and metabolic features of walking gait between healthy and stroke subjects is heavily reported \citep{Chen2005, Balaban2014, Reisman2009, Ellis2013, Detrembleur2003}. 
The priority of current rehabilitation protocols is to obtain walking independence as quickly as possible, at the cost of the functional restoration of the paretic limb \citep{Kitago2013}. 
This is often measured through walking speed, as it is associated with many temporospatial parameters of gait \citep{Balaban2014}. 
While increases in walking speed are linked to walking independence, short term improvements are often the result of  compensatory strategies \citep{Chen2005}. 
Common solutions such as ankle-foot orthoses, while they aid the paretic limb in ground clearance, have negative effects on propulsion and gait adaptability, which increase the energy cost of walking \citep{VanSwigchem2014,Vistamehr2014,Wutzke2012}.\par 

% energy costs [from Lou's AHA grant]
Higher energy costs of walking in both older adults and populations with neuromotor walking deficits have been found to be a primary contributor to physical inactivity, as shown by the energy cost of walking being up to twice as high for those post-stroke compared to healthy walkers \citep{Franceschini2013,Lapointe2001,Moore2010,Wert2013, Detrembleur2003}. 
Even the energy cost of walking of community-dwelling poststroke individuals is double of what is predictive of future mobility decline \citep{Awad2016}. 
When made to walk at more normal walking speeds, poststroke individuals consume less energy than at their normal speeds; however, this level is still highly inefficient due to reliance on gait compensations \citep{Hsiao2016a,Reisman2009, Detrembleur2003, Bowden2013}. 
As gait symmetry has been shown to be an important part in minimizing the energy cost of walking in healthy subjects, reducing asymmetries in stroke gait has shown to produce faster, less costly walking \citep{Ellis2013, Awad2015a}. \par

% finish this question up, tie it to the next one
Propulsion is an important function of human walking gait; despite this, conventional rehabilitation efforts have been unable to restore patient's ability to generate propulsion symmetrically from each limb. 
The development and study of interventions targeting propulsion function is an active area of research \citep{Awad2017c,Browne2019}. 
However, there remains a need for measurement instruments, accessible to clinicians, that allow for the clinical management of individual limb propulsion deficits.
Wearable sensors provide an opportunity  to bridge this gap. \par

% % % % % % % % % % % % % % % % % % % % % % % % % % % % % % % % % % % % % % % % % % % % % % % % % % % % %

\section{Wearable Sensors}
% general intro
For decades, sensors have been used to study the biomechanics of gait \citep{Cutting1977}. 
Paralleled with the boom in sensing technology, many different technologies have been used in gait detection, such as video-based \citep{Wang2003}; floor-based \citep{Vera-Rodriguez2012}; and wearable sensor based technology \citep{Gafurov2007}. 
Of these, the development of wearable sensors for the study and advancement of gait interventions is a highly active area of research \citep{Stanton2017, Sprager2015, Taborri2016}. \par 

% not good sensors 
Some sensors - such as footswitches, foot pressure insoles and electromyography (EMG) - have applications in the analysis of gait; however, they also have major limitations. 
Footswitches are considered the gold standard for gait phase detection among wearable sensors, as they directly detect foot contact with the ground; as such, they are often used to validate other types of sensors \citep{Abaid2013, Mannini2012, Taborri2014}. 
As they can only detect foot contact with the ground, footswitches are unable to detect anything during the swing phase and cannot measure forces during the gait cycle \citep{Pappas2004}. 
Patients with impaired gait may also produce issues in accuracy and reliability as their gait patterns can be irregular \citep{Aminian2002}. \par

Foot pressure insoles improve on footswitches, as they can record contact of the full foot with the ground along with ground reaction forces. 
Combined with machine learning, this allows for more time sensitive gait detection and the discrimination of different parts of the swing phase \citep{Crea2012, Jacobs2015, Catalfamo2008}. 
However, foot pressure insoles have limitations of their own, such as compromising foot-ground interaction; repeated mechanical stress leading to sensor wear out; and participant discomfort \citep{Ancillao2018}. 
EMG signals are used in gait phase detection due to repeatable muscle activity in the lower body \citep{Hof2002}, and in studying muscle coordination and synergies \citep{Clark2010, Steele2015}. Drawbacks of EMGs include higher complexity in acquiring and processing data, as well as the price of equipment.\par

% good sensors
Inertial sensors allow for the study of walking biomechanics beyond gait phases, such as spatiotemporal (\textit{e.g.} step length, cadence) and kinematic parameters (\textit{e.g.} joint angles). 
Using kinematic parameters, ground reaction forces can be estimated; these are of particular importance, as they allow for the study of kinetic forces during walking and the calculation of internal joint forces and moments through inverse dynamics \citep{Ancillao2018}. 
Examples of inertial sensors include accelerometers, gyroscopes and inertial measurement units (IMUs). \par

Accelerometers are sensors that measure linear acceleration of a body in its own instantaneous rest frame. 
As they are small, inexpensive and readily available \citep{Kavanagh2008}, accelerometers are a common solution in gait analysis, using a variety placements and number of sensors \citep{Taborri2016}. 
Multiple studies have investigated various positioning of sensors on the body with the goal of studying segmentation of the gait cycle and walking incline \citep{Mijailoviu2009, Rueterbories2014, Selles2005}. 
For this, peaks at the start and end of the stance phase in the anterior-posterior direction allow for the detection of two gait phases \citep{Taborri2016}. 
Accelerometers have also been shown to provide accurate spatiotemporal parameters for stroke subjects, both in the laboratory and community \citep{Moore2017}. 
In these and other applications, signal shaping procedures are necessary to improve performance; a machine learning approach is not mandatory, however. 
Despite their many applications, accelerometers are imperfect: gravity must be compensated for when calculating body segment acceleration; short term drift error occurs when calculating linear velocity and position due to numerical integration of the signal and high frequency noise, resulting in a process that is computationally expensive; and it is required to place the sensors correctly on the body segment, along with a calibration procedure. \par

Gyroscopes are sensors used to measure angular velocity and are a popular solution for gait detection as they are not influenced by gravity or vibrations due to heel strike \citep{Taborri2015, Mayagoitia2002}. 
Just as with accelerometers, sensor placement has been extensively studied \citep{Catalfamo2010, Mannini2011, Mannini2012, Taborri2015a}. 
Based on this, gyroscopes can detect up to six separate gait phases when combined with machine learning approaches, even during daily activity. 
However, also like accelerometers, they are affected by drift errors due to numerical integrations to compute angular position; this drift occurs more slowly, however. \par

Combining multiple different sensors - such as an accelerometer, gyroscope and magnetometer - IMUs are able to compensate for the drawbacks of each using sensor fusion techniques such as Kalman filtering. For example, they address drift error found in both accelerometers and gyroscopes, and use magnetometers as a heading reference in the calculation of global variables; this allows for more robust analyses and the computation of spatiotemporal parameters and kinematic variables such as joint angles \citep{Evans2014, Donath2016, Dejnabadi2005}. 
IMUs can allow for the estimation of ground reaction forces through their measurement of kinematic data, using either biomechanical models \citep{Aurbach2017a, Karatsidis2016, Ohtaki2001} or machine learning \citep{Guo2017, Wouda2018}. 
From these approaches, vertical ground reaction forces can be accurately estimated during the single stance phase; calculating the forces for each limb during double support is more difficult. 
While there are methods of estimating anterior-posterior ground reaction forces, they face difficulties with neurologically impaired populations, as current methods depend on assumptions of healthy, consistent walking patterns \citep{Ancillao2018, Shahabpoor2018}. \par

% IMUs vs gold standard
Specialized laboratory equipment, such as force plates and optical motion tracking systems, are the gold standard for measuring kinetics and kinematics in both healthy \citep{Franz2013, Franz2014} and neurologically impaired walking \citep{Bethoux2011, Bowden2006, Farris2015, Halliday1998, Hass2012, Jonkers2009, Martin2002, Turns2007}. 
Located either in the ground or as part of an instrumented treadmill, force plates measure kinetic forces using multicomponent load cells. 
Combined with optical motion tracking - which uses reflective markers to track body segments - joint forces and moments can be obtained through the use of inverse dynamics. 
While their level of accuracy is unparalleled, there are difficulties with these instruments. 
In regards to in-ground force plates, several trials are needed to obtain an adequate amount of data as often only a few strides can be obtained due to the number of force plates and inconsistent foot strikes; this limits the number of consecutive gait cycles that can be reviewed. 
Instrumented treadmills solve the problem of limited gait cycles, but introduce the difficulty of unrealistically consistent gait speed and the differences in biomechanics between overground and treadmill walking. 
These, along with optical motion capture, require both large, dedicated spaces and are prohibitively expensive. Due to this, they are often found only in laboratory settings and are inaccessible to most clinicians.\par

Wearable, inertial sensors provide a way to solve these complications presented by force plates and optical motion tracking. 
Many studies have shown that inertial sensors can produce data of comparable quality to these gold standards, such as: spatiotemporal parameters at different speeds and slopes, in the community and across different populations \citep{Donath2016, Moore2017, Trojaniello2014}; kinematic parameters at different speeds \citep{Mayagoitia2002}; quantification of gait symmetry \citep{Zhang2018} and estimations of kinetic parameters such as ground reaction forces \citep{Pieper2019, Karatsidis2019, Ryu2018, Lim2019}. 
Of all the benefits of wearable sensors, the ability to study gait parameters outside the laboratory may be the most promising \citep{Bejarano2015, Boutaayamou2015, Miyazaki2019, Peruzzi2011, Seel2014, Yang2013}. 
As they are inexpensive and do not require large amounts of dedicated space, wearable sensors can help bridge the gap between the laboratory and clinic. \par

% IMU drawbacks
Wearable sensors are not without their drawbacks when compared to specialized laboratory equipment. State-of-the-art technology is often more accurate and precise, as they are the gold standard that others are compared to. For example, force plates can measure ground reaction forces directly, while any kinetic measure taken from IMUs will be estimations. Wearable, inertial sensors also lack the ability to find an absolute heading and accurate joint angle measurements without a magnetometer; the latter can be dealt with through the assumption of an always-zero yaw, limiting joint angle measurements to two dimensions. \par


% biofeedback
One potential use for wearable sensors in a clinical setting is biofeedback. 
Biofeedback is the use of technology to provide information to the learner that is not consciously available by transforming biological signals into an output they can understand and has been found to improve walking and balance when compared to usual therapy \citep{Stanton2017}. 
Training of this kind is advantageous to conventional therapy due to it's targeted intervention, sensitive and immediate feedback, motivating effect due to game-like features and the ability to perform supervised exercises from home \citep{Horak2015, Zijlstra2010}. \par

Information such as the kinetics, kinematics and muscle activation of walking can be delivered through the visual \citep{Karatsidis2018}, auditory and tactile senses \citep{Schenck2019}. 
Of these, visual feedback is the most commonly reported in the literature, as is information on kinematic parameters \citep{VanGelder2018}. 
It has been found that providing feedback using multiple senses results in more positive outcomes than separate modes of feedback \citep{Sigrist2013}; in addition, combined visual and sensory feedback has been shown to outperform each individually \citep{Yen2014}. 
Biofeedback on muscle activity lags behind those of kinematic, kinetic and spatial-temporal parameters \citep{Tate2010}. 
In regards to ground reaction forces, kinetic biofeedback increased both propulsive forces and muscle activation during push-off compared to feedback on muscle activity \citep{Franz2014}; however, there is no consensus of any form of feedback being superior in regards to ground reaction forces  \citep{Agresta2015}.\par

Biofeedback may provide an avenue to assist in the rehabilitation of walking after stroke, particularly in regards to propulsion \citep{Franz2014}. 
An increase in peak anterior-posterior ground reaction forces in a single, unilateral leg after a period of biofeedback training encourages the use of such training in post-stroke individuals \citep{Schenck2017}. 
Using biofeedback focused on increasing anterior-posterior ground reaction forces has shown to produce positive changes in trailing limb angle and ankle plantarflexor moment and push off in those post-stroke \citep{Genthe2018}; both of these parameters encourage the retraining of a more normal, economic gait pattern. 
This could also occur with the reduction of hip power output during walking, known as distal-to-proximal redistribution, that is common in post-stroke individuals \citep{Browne2019}. 
Along with a myriad of other positive changes, such as peak ankle power and stride length \citep{Jonsdottir2010}, biofeedback may provide an opportunity of effective gait retraining in those post-stroke. \par

Unfortunately, the vast majority of published literature on biofeedback is in a laboratory setting  \citep{VanGelder2018} and is focused on upper limb, cognitive and balance rehabilitation, particularly though the use of virtual reality and  video game applications \citep{Gamito2017, Li2016, Perez-Marcos2017, Laver2017}. 
There remains a need for walking biofeedback interventions using wearable sensors that can be translated to a clinical setting.
There is a potential tool in using wearable sensors for walking biofeedback in machine learning. \par 


% % % % % % % % % % % % % % % % % % % % % % % % % % % % % % % % % % % % % % % % % % % % % % % % % % % % % %

\section{Statistical and Machine Learning}

% general intro
Machine learning can be defined as programming a computer to adapt or change its outputs based on certain inputs, where the goal is to have those outputs reflect the correct ones \citep{Marsland2011}. 
By this definition, the algorithms produced are said to be learning if they adapt so that their performance improves - in short, they learn from the data \citep{Hastie2009}. \par 
A common framework of machine learning is statistical learning, where  statistics and functional analysis is used to find a predictive function based on the data \citep{Hastie2009}.

% classification
Classification consists of taking a vector of inputs and deciding which of a number of classes they belong to \citep{Marsland2011, Bishop2006}. 
In most cases, the classes are taken to be separate, so that each input belongs to a single class and the set of classes accounts for all of the possible output space; this is not always realistic, and at times fuzzy classifiers or output histograms are used to solve this problem \citep{Marsland2011, Bishop2006, Goodfellow2016}. 
Though there are different ways of finding a solution, they all aim to do the same thing: determine different decision regions separated by decision boundaries, the most simple example being linear boundaries \citep{Marsland2011, Bishop2006, Hastie2009}. \par

% regression
Regression is a statistical solution that aims to fit a linear mathematical function that passes as close as possible to all the data points \citep{Marsland2011}. 
Because they are linear, the change in the dependent variable is predicted by the independent variable in a constant rate.
Whether done by directly constructing an appropriate function or modeling the predictive distribution, linear models for regression are simple, and are often sufficient to gain insight into how the inputs affect the output in lower dimensional spaces \citep{Bishop2006}. 
Though this can be useful in higher dimensional spaces, other methods may be better suited for that purpose; however, it does form the foundation of more sophisticated models, including nonlinear ones \citep{Marsland2011, Bishop2006, Hastie2009}.  \par  

% Multiple and multioutput
Multiple or multiregressor regression is an extension of simple linear regression where multiple independent variables are used to predict the outcome variable \citep{Marsland2011}.
This is done by modeling the relationship between dependent and independent variables simultaneously. Multioutput regression extends this to cases where there are multiple output variables, allowing one to identify the best subset of predictor variables that explain the variance in each output \citep{Marsland2011}.

% Forward stepwise regression
A common method of building a multiregressor model is forward stepwise regression, an iterative algorithm that selects variables based on their contribution to the predictive power of the model by minimizing the loss function, typically mean squared error (MSE) \citep{Hastie2009}\.
This is done by starting with a model with no regressors and sequentially adding variables based on their correlation with the current residuals. \par

% supervised learning intro
Machine learning using statistical analysis can come in different forms. 
In supervised learning, quantitative or categorical outcome measurements are based on individual measurable properties or characteristics of what is being observed, referred to as features \citep{Bishop2006}. 
In turn, they are developed using a training set of data with the correct responses to allow the algorithm to generalize its response to all possible inputs \citep{Hastie2009, Marsland2011}. 
This generalization is important: the ability to produce sensible outputs for inputs which were not met during learning - and dealing with the noise found in real-world data - is the central goal of  machine learning \citep{Marsland2011, Bishop2006}. 
Two of the most common examples of supervised learning are regression and classification. \par 

% other forms of learning
In unsupervised learning, the correct responses are not provided; only the features are observed and the algorithm is tasked with describing how the data are organized or grouped based on similarities \citep{Hastie2009, Marsland2011}. 
This can take the form of finding groups of similar examples in the data, known as clustering; determining the distribution of data within an input space, called density estimation; or to simplify data from a high dimensional space down to two or three dimensions, referred to as dimensionality reduction or component analysis \citep{Bishop2006, Goodfellow2016}. 
Reinforcement learning falls somewhere between supervised and unsupervised learning: the algorithm is told when the answer is wrong but is not told the correct answer, requiring it to explore and try different possibilities \citep{Marsland2011}. 
Here, there should be a balance between exploration (trying new actions to test their effectiveness) and exploitation (making use of actions that are known to be successful) as a focus on either will lead to poor results \citep{Bishop2006}. 
Finally, evolutionary learning takes inspiration from biological evolution, using an idea of fitness to determine the viability of the current solution \citep{Marsland2011}. \par

Although machine learning can take many forms, there are key ideas that provide the foundation to common problems and solutions.
Regardless of the machine learning approach taken, a training set will be used to tune the parameters of an adaptive model \citep{Bishop2006}.
This set is made up of data that has been inspected in advanced in order to train the model.
Once trained, the performance of the model can be tested on a test set in order to determine generalization of the model.
As the input vectors of the training data can only contain a small fraction of all possible inputs, the generalization of the model is a central goal. \par  

A common method to evaluate performance and generalization of a machine learning model is cross-validation \citep{Brunton2021}.
This method allows the model to perform on unseen data by splitting the dataset into testing and training sets multiple times, with performance metrics being averaged across each split. 
The most robust form of cross-validation is leave-one-out (LOO) cross-validation, where a single observation is used for testing and all others are used in the training set, allowing for each observation to be used in both the training and testing sets \citep{Brunton2021}. 
As this results in k-1 models being trained, LOO cross-validation is computationally expensive; however, it is often the most complete form of cross-validation as it provides the most accurate estimation of model performance. 

For classification, accuracy, sensitivity and specificity are commonly used to determine performance:

% Equation 7, Begg2005
\begin{align}\label{Gen1} % set label for equation
	 & Accuracy =  \frac{TP + TN}{TP + FP + TN + FN} \times 100\% \nonumber \\
	 & Sensitivity = \frac{TP}{TP + FN} \times 100\%                        \\
	 & Specificity = \frac{TN}{TN + FP} \times 100\% \nonumber
\end{align}

\noindent where TP is the number of true positives, TN is the number of true negatives, FP is the number of false positives and FN is the number of false negatives \citep{Begg2005}. 
Studies that focused on real-time estimations also compared the computational load of their models \citep{Taborri2014}. 

Performance of a regression analysis can determined using root mean squared error (RMSE), mean absolute percent error (MAPE), \(R^{2}\), and adjusted  \(R^{2}\)  \citep{Hastie2009}. 

% Equation , common knowledge
\begin{align}\label{Gen2} % set label for equation
	 & RMSE = \dfrac{\sqrt{\sum_{i=1}^{N} (Predicted_{i} - Actual_{i})^{2}}}{N}                               \\
	 & MAPE = \frac{1}{n} \sum_{i=1}^{n} \left| \frac{y_i - \hat{y}_i}{y_i} \right| \times 100                \\
	% & L_{Huber}(y,\hat{y}) = \begin{cases}
	% \frac{1}{2} (y - \hat{y})^2                 & \text{if } |y - \hat{y}| \leq \delta               \\
	% \delta (|y - \hat{y}| - \frac{1}{2} \delta) & \text{if } |y - \hat{y}| > \delta \text{otherwise}
	% \end{cases} \\
	 & R^{2} = 1 - \dfrac{\sum_{i=1}^{N} (Y_{i} - \hat{Y}_{i})^{2}}{\sum_{i=1}^{N} (Y_{i} - \bar{Y}_{i})^{2}} \\
	 & R_{adj}^{2} = 1 - \dfrac{(1-R^{2})(n-1)}{n-k-1}
\end{align}

RMSE measures the average of the squared difference between the predictions and the ground truth, allowing for the detection of outliers; it also is in the same units as the target variable, making it more interpretable and easier to compare across models. MAPE measures the average absolute difference between the predicted and actual target values, allowing for a clear understanding of how the model has performed in relation to the true values. Finally, \(R^{2}\) uses the regression and total sum of squares to determine the variance between the model and a linear model; \(R_{adj}^{2}\) adds a penalty for the total number of observations, allowing for a better comparison between models with a different amount of variables.	 

When studying bipedal locomotion biomechanics, three methods are common in the literature: support vector machines \citep{Tahir2012, Begg2005a, Begg2005, Mannini2016}, neural networks  \citep{Tahir2012, Kaczmarczyk2009, Alaqtash2011} and Hidden Markov Models \citep{Mannini2016, Cuzzolin2017}.  \par
%
% % SVM
When studying bipedal locomotion biomechanics, one common machine learning method is support vector machines \citep{Tahir2012, Begg2005a, Begg2005, Mannini2016}.
Using support vector machines (SVMs), known as support vector regression (SVR) when used in regression analysis, a feature vector with dimension \(m\), \citep{Begg2005a} describes a hyperplane in \(m\) dimensional space is found that linearly separates the two classes on either side of the hyperplane, or decision surface, whose equation is:

% Equation 1, Begg2005a
\begin{equation}\label{SVM1} % set label for equation
	\vec{w}^{T}\vec{x} + b = 0 
\end{equation}

\noindent where \(\vec{w}\) is the adjustable weight vector and  \(b\) is the hyperplane bias. With a classification output of \(\{+1, -1\}\), the linearly separable case can be represented as:

% Equation 2, Begg2005a
\begin{gather}\label{SVM2} % set label for equation
	\vec{w}^{T}\vec{x} + b \leq 0 \quad for \quad d_{i} = -1 \nonumber \\
	\vec{w}^{T}\vec{x} + b > 0 \quad for \quad d_{i} = +1.
\end{gather}

As mentioned previously, in realistic situations - such as bipedal locomotion biomechanics - datasets might not be linearly separable. This has two potential solutions: applying nonlinear transforms, such as binning, or using kernels \citep{Begg2005a}.
In binning, values of a continuous variable are put into bins with values around it; this may address issues such as missing values, presence of outliers, statistical noise and data scaling. 
Unfortunately, this can be unmanageable with larger values of \(m\) \citep{Begg2005a}. 
This leads to the use of kernels, which use kernel functions (such as polynomials, sigmoid functions and radial basis functions) to map the non-linear observations into a higher dimensional space where they are separable \citep{Marsland2011, Bishop2006, Hastie2009}. 
These are similar to m-dimensional vectors, with an exception: the inner products required by SVM are computed without explicitly building the high-dimensional representations. 
How well a hyperplane fits in feature space is measured as the distance between the hyperplane and the support vectors \citep{Begg2005a}. 
An advantage of SVMs is its low computational cost; the solution of the optimization is straightforward due to the objective function being convex, and the number of basis functions in the model is much smaller than the number of training points \citep{Bishop2006}. \par

% % NN	
An alternative to SVMs is neural networks (NNs). 
Started as an attempt to mathematically model information processing in biological systems, NNs can be thought of as a system of neurons that take inputs and produce outputs, often represented by a neural diagram as in Figure \ref{fig:NN1} \citep{Bishop2006, Hastie2009}.
NNs that contain more than a single hidden layer are often called deep neural networks. \par

In this approach, the number of basis functions are fixed in advance and allowed to be adaptive during training \citep{Bishop2006}. 
With this, it has come to include a large amount of models and learning methods. At their core, they remain fairly simple: they are nonlinear statistical models \citep{Hastie2009}. 
In a \(K\)-class classification, there are \(K\) target measurements (\(Y_{k}\), \(k=1\),\(...\),\(K\)), each coded as a 0 or 1 variable for the \(k\)th class. Linear combinations of inputs for the derived features (\(Z_{m}\)) are again linearly combined to create the target (\(Y_{k}\)), as shown in \citep{Hastie2009}:

% Equation 3, Hastie2009
\begin{gather}\label{NN1} % set label for equation
	Z_{m} = \sigma(\alpha_{0m} + \alpha^{T}_{m}X), m = 1, ... M, \nonumber \\
	T_{k} = \beta_{k} + \beta_{k}^{T}Z, k = 1, ... K,  \\
	f_{k}(X) = g_{k}(T), k = 1,... K, \nonumber
\end{gather}

\noindent where \(Z = (Z_{1}, Z_{2},...Z_{M})\) and \(T = (T_{1}, T_{2},...T_{K})\). 
The activation function \(\sigma(v)\) is often chosen to be the sigmoid function \(\sigma(v) = 1/(1+e^
{-v})\); alternatively, Gaussian radial basis functions are also used \citep{Hastie2009}. \par

\begin{figure}[t]
	\includegraphics{NN_figure}
	\centering
	\caption{Example of a single hidden layer, feed-forward neural network \citep{Hastie2009}. The top layer (\(Y\)) are in inputs; the bottom layer (\(X\)) are the outputs; and the middle layer (\(Z\)) is the hidden layer; each layer can have any number of neurons.}
	\centering
	\label{fig:NN1}
	\vspace{0.5cm} % tasteful amount of space
\end{figure}

% HMMs
In probability theory, a chain is a sequence of possible states whose probability is a function of the previous states. Markov chains use the Markov property: the probability at any given time (\(t\)) is only dependent on the time state immediately before it (\(t-1\)) \citep{Marsland2011}. 
These set of states are linked by the likelihood of moving from the current state to another state, or transition probabilities \citep{Marsland2011, Bishop2006}. 


As described in \citep{Marsland2011}, a HMM is made up of three parameters: transition probabilities (\(a_{i,j}\)), observation probabilities (\(b_{j}(o_{k})\)) and the probability at starting at any state (\(\pi_{i}\)). Defining a sequence of probabilities (\(O\)) and a sequence of possible states (\(\Omega\)),

% Equaption 4, Marsland2011
\begin{equation}\label{HMM1} % set label for equation
	P(O) =  \sum_{r=1}^{R}P(O|\Omega_{r})P(\Omega_{r}) 
\end{equation}

\noindent where \(r\) describes a sequence of states. Using the Markov property, we can obtain:

% Equation 4, Marsland2011
\begin{gather}\label{HMM2} % set label for equation
	P(\Omega_{r}) =  \prod_{t=1}^{T}P(\omega_{j}(t)|\omega_{j}(t-1)) = \prod_{t=1}^{T}a_{i,j} \\
	P(O|\Omega_{r}) =  \prod_{t=1}^{T}P(o_{k}(t)|\omega_{j}(t)) = \prod_{t=1}^{T}(b_{j}(o_{k}).
\end{gather}

Following this, equation \ref{HMM1} can be rewritten as:

% Equaption 5, Marsland2011
\begin{equation}\label{HMM3} % set label for equation
	P(O) =  \sum_{r=1}^{R}\prod_{t=1}^{T}(b_{j}(o_{k})a_{i,j}. 
\end{equation}

While this outcome is simplified, it poses a critical issue: assuming there are \(N\) hidden states and \(N^{T}\) possible sequences, the computational cost of calculating all \(T\) probabilities would be far too great (\(O(N^{T}T)\)). \par

However, as stated previously, the Markov property states that the probability of each state depends only on the current and previous state (\(o(t),\omega(t),\omega(t-1)\)). 
This allows for the computation of \(P(O)\) to be performed one step at a time. 
Introducing \(\alpha_{i}(t)\) as the probability that the state is \(\omega_{i}\) at time \(t\), and that the first (\(t-1\)) steps all matched the observations \(o(t)\):

% Equaption 6, Marsland2011
\begin{gather}\label{HMM4} % set label for equation
	\alpha_{j}(t) = \begin{cases} 
		0                                               & t = 0, j \neq initial state \\
		1                                               & t = 0, j = initial state    \\
		\sum_{i} \alpha_{i} (t-1) a_{i,j} b_{j} (o_{t}) & otherwise. 
	\end{cases}
\end{gather}

This requires only the probability of observation \(o(t)\) to contribute to the sum, allowing the computation of \(P(O)\) to be a much improved \(O(N^{2}T)\). 
Continuing, if \(a_{i,j}\) is the probability of transitioning from state \(i\) to state \(j\), \(a_{i,t}\) is then the probability of obtaining the observation sequence until a time \(t\) and being in the state \(i\) at time \(t\). 
As this is conditioned on the model and the observations, the probability of moving forward through the entire observation sequence is given as: 

% Equaption 7, Marsland2011
\begin{equation}\label{HMM5} % set label for equation
	\sum_{i=1}^{N}\alpha_{i,T}.
\end{equation}

Moving backwards through the sequence requires a similar algorithm based on the transmission and observation probability matrices: 

% Equaption 8, Marsland2011
\begin{equation}\label{HMM6} % set label for equation
	\beta_{i,t} =  \sum_{j=1}^{N} a_{i,j} b_{j} (o_{t+1}) \beta_{j,t+1}. 
\end{equation}


% sensor fusion
An alternative use of machine learning in the study of bipedal locomotion is the analysis and estimation of gait parameters using multiple sensors and a variety of machine learning methods \citep{Caldas2017, Figueiredo2018}. 
Termed "sensor fusion," this method combines data from multiple sources to produce outputs.
Regression is used to estimate outputs such as walking speed \citep{McGinnis2017} and ground reaction forces \citep{KieronJie-Han2018}; classification is the method of choice for gait phase detection \citep{Taborri2015a, Taborri2014}, gait event detection \citep{Novak2013, Mannini2014}, gait symmetry \citep{Yang2013} and other spatiotemporal parameters \citep{Shetty2016}.   
This is an emerging research area with positive promise; however, it does suffer from a lack of standardization in both performing and reporting evaluations \citep{Caldas2017, Figueiredo2018}.


% how to apply machine learning
With a large number of possible solutions, it is important to follow a set of guidelines to properly select, apply and evaluate machine learning algorithms to a particular problem as proposed by  \citep{Marsland2011}. 
First and foremost is data collection and preparation. 
When asking a new question, it is important to determine what data will be collected; this can be done by collecting a moderate sized data set with features that are believed to be useful and experimenting to find which are of use. 
In doing this, it is important that the data is clean, meaning it does not have significant errors or missing data. When using supervised learning, target data is also needed to train the model, often requiring involvement of experts in the field and significant investments of time. 
Also to be considered is the quantity of data; significant amounts are needed to train an algorithm, but computational costs must also be considered. This step is crucial and can highly affect the potential solutions. \par 

Next, feature selection should be considered. 
Prior knowledge of the problem and the data is important when identifying features that will be the most useful. 
After the data set and features have been selected, knowledge of the underlying principles of  different types of algorithms and their uses is paramount for selecting the optimum algorithm for the problem. 
When applicable, manual selection of parameters should result from prior knowledge and experimentation to identify appropriate values. 
Careful selection of the training data set and parameters must also be considered. 
Finally, the algorithm must be tested for accuracy on data that it was not trained on; this may include comparisons to human experts in the field. \par  

\begingroup
\renewcommand{\arraystretch}{1.5} % change space between rows
\setlength{\arrayrulewidth}{1mm} % change width of lines

\begin{table}[ht]
	\centering
	\resizebox{\textwidth}{!}{ % make table fit on page
		\begin{tabular}{l l l l} % number of columns and alignment
			
			Authors                 & Method   & Data Types                         & Outcome                       \\ % column headers
			\hline % line under headers
			
			\citep{Tahir2012}       & SVM, NN  & spatiotemporal, kinetic, kinematic & healthy and pathological      \\ 
			\citep{Kaczmarczyk2009} & NN       & kinematic, qualitative             & pathological classification   \\ 
			\citep{Begg2005}        & SVM      & kinetic, kinematic                 & young and ageing              \\ 
			\citep{Alaqtash2011}    & NN       & kinetic                            & pathological classification   \\ 
			\citep{Mannini2016}     & SVM, HMM & spatiotemporal, kinematic          & healthy and pathological gait \\ 
			\citep{Cuzzolin2017}    & HMM      & kinematic                          & healthy and pathological gait \\ 
			
		\end{tabular}}
	
	\caption{Studies using classification, in order of dataset size.}
	\label{table:1}
	\vspace{0.5cm} % tasteful amount of space
\end{table}

\begin{table}[h]
	\centering
	\resizebox{\textwidth}{!}{ % make table fit on page
		\begin{tabular}{l l l l} % number of columns and alignment
			
			Authors                   & Method  & Data Types                         & Sensors             \\ % column headers
			\hline % line under headers
			
			\citep{McGinnis2017}      & SVM     & kinematic                          & accelerometer       \\
			\citep{Farah2017}         & SVM     & kinetic, kinematic                 & optical, forceplate \\
			\citep{Taborri2015a}      & HMM     & kinematic                          & IMU                 \\
			\citep{Taborri2014}       & HMM     & kinematic                          & IMU                 \\
			\citep{Novak2013}         & SVM     & spatiotemporal, kinetic, kinematic & IMU, insole         \\
			\citep{Potluri2019}       & SVM, NN & spatiotemporal, kinematic          & IMU, insole         \\
			\citep{Mannini2014}       & HMM     & kinematic                          & gyroscope           \\
			\citep{KieronJie-Han2018} & NN      & kinematics                         & accelerometer       \\
			
			
		\end{tabular}}
	
	\caption{Studies using sensor fusion, in order of dataset size.}
	\label{table:2}
	\vspace{0.5cm} % tasteful amount of space
\end{table}

\endgroup

Across the literature (Tables \ref{table:1}, \ref{table:2}), there are varying practices with regards to the above guideline, as well as various levels of reporting.
In the study of bipedal locomotion in healthy and neurologically impaired populations, the large majority of studies use time series data such as saptiotemporal, kinetic and kinematic data (Tables \ref{table:1}, \ref{table:2}); some also used point metrics of time series data such as maximum and minimum forces and joint angles \citep{Tahir2012, Begg2005}. 
All these are most often collected via optical motion tracking, force plates or wearable sensors. 
Data preparation often takes the form of normalization, such as ground reaction forces normalized to body weight \citep{Alaqtash2011} and intra and inter group data normalization \citep{Tahir2012}. 
In relevant studies, gait phase segmentation was manually computed to provide added features \citep{McGinnis2017, Taborri2015a}. 
When multiple sensors were used, sensor fusion techniques such as Kalman filters (for reducing drift) and Butterworth filters \citep{Novak2013} were applied.

Many studies do not report how features were extracted or chosen \citep{Begg2005, McGinnis2017}; some use content expertise to drive their decisions \citep{Alaqtash2011}; others use methods such as HMMs \citep{Mannini2016}; alternatively, the features at times are chosen "arbitrarily" \citep{Kaczmarczyk2009}. 
Model selection was similar across both outcomes (Tables \ref{table:1}, \ref{table:2}): predominantly SVMs, NNs and HMM were used, with other models such as nearest neighbor classifiers, principal component analysis and cluster analysis used as points of comparison. 

How the models were trained also varied. A number of studies used cross-validation due to its usefulness with limited data sets \citep{Tahir2012, McGinnis2017,
	Taborri2014, Novak2013}; others used supervised learning approaches to train the models on a subset of data \citep{Mannini2016}. 
Evaluation of the models were performed by comparing them to a gold standard, that being technology \citep{Begg2005, Mannini2016, Taborri2015a} or an
external standard \citep{Kaczmarczyk2009}. \par

In the field of bipedal locomotion biomechanics, machine learning is quickly emerging as a hot topic. 
However, the use of machine learning and wearable sensors to estimate anterior-posterior ground reaction forces - and in turn, propulsion -  has not been studied. 
There remains a need for literature dedicated to studying propulsion and to closing the gap between laboratory-based measurements of propulsion and those accessible to clinicians. Machine learning, combined with wearable sensors, may be able to close that gap.

% % % % % % % % % % % % % % % % % % % % % % % % % % % % % % % % % % % % % % % % % % % % % % % % % % % % % %

\section{Research Questions}
The 6-Minute Walk Test (6MWT) is a popular outcome measure used to evaluate long distance walking function after stroke; however, the most often outcome measured, total distance walked, does not directly speak to the underlying neuromotor impairment.
In order to promote individualized rehabilitation interventions that can target specific propulsion deficits, the question of what contributes biomechanically to reduced 6MWT performance is necessary.
Biomechanical variables such as the peak propulsion force (the peak of the anterior ground reaction force) and the propulsion impulse (the delivered force of the anterior ground reaction force) have been shown to contribute to walking function after stroke \citep{Awad2015}.
However, these magnitude-based metrics do not account for temporal aspects of propulsion function, such as propulsion peak timing \citep{Kuhman2019}; as they are propulsion-based, they also do not take into consideration braking magnitude and timing.
There remains a need to determine if the timing and braking aspects of poststroke walking gait contribute to long-distance walking function. \par
As such, the first aim  of this dissertation is to \textbf{evaluate the effects of propulsion metrics, beyond peak propulsion magnitude and propulsion impulse, on long distance walking function}.
The hypotheses for this aim are: 1) other propulsion metrics will contribute to the relationship of 6MWT performance, and peak propulsion and propulsion impulse, respectively; and 2) Propulsion timing metrics will moderate the relationship between 6MWT performance, and peak propulsion and propulsion impulse, respectively. \par


While the study and development of interventions targeting propulsion function after neurological injury is a highly active area of research \citep{Browne2017, Boutaayamou2015, Miyazaki2019, Awad2017c, Genthe2018, Kesar2009, McCain2019, Penke2019, Phadke2012, Takahashi2015}, the clinical use of these approaches is limited due to reduced access to the gold standard technology used to directly measure anterior-posterior ground reaction forces (APGRFs) \citep{Farris2015, Bowden2006, Jonkers2009, Turns2007, Bethoux2011, Franz2014, Franz2013, Martin2002}.
As long as the instruments used to measure propulsion remain inaccessible to clinicians, interventions targeting specific deficits in walking function will continue to be out of reach. \par
Wearable sensors are a potential solution to this issue: they have been used to collect a number of gait measurements outside the laboratory \citep{Boutaayamou2015, Miyazaki2019, Peruzzi2011, Seel2014, Yang2013}, and have proven effective at indirect measurements of ground reaction forces during walking \citep{Karatsidis2019, Ryu2018, Lim2019, Shahabpoor2018}.
Specifically, inertial measurement units (IMUs) have recently shown to be useful in measuring propulsion metrics, such as peak propulsion magnitude and timing, and propulsion impulse, in hemiparetic walking \citep{Pieper2019, Revi2020}.
With the help of machine learning, IMUs may be the key to extending the study of paretic limb propulsion to the clinic. \par
As such, the second aim of this dissertation is to \textbf{validate the accuracy of an IMU based algorithm in estimating propulsion metrics versus the gold standard}. 
The hypotheses for this aim are: 1) this IMU based machine learning algorithm will accurately estimate all propulsion metrics against the gold standard; and 2) this IMU based machine learning algorithm will outperform the published literature in estimating all propulsion metrics. 
The final aim will be to \textbf{validate the accuracy of an IMU based algorithm in estimating the APGRF time series curve versus the gold standard}.
The hypotheses for this aim are: 1) this IMU based machine learning algorithm will accurately estimate the APGRF time series curve versus the gold standard; and 2) this IMU based machine learning algorithm will outperform the published literature in estimating the APGRF time series curve. \par
