% ABSTRACT

% Have you ever wondered why this is called an \emph{abstract}? Weird thing is
% that its legal to cite the abstract of a dissertation alone, apart from the
% rest of the manuscript.

The efficiency of human walking gait is an important factor of what makes walking such a vital part of human daily living.
However, this efficiency, particularly that of propulsion forces, is inhibited in populations with neuromotor walking deficits, including stroke.
The ability to measure propulsion forces accurately involves the use of large and expensive equipment that is often not available for clinicians to use.
With wearable sensors and machine learning, there presents an opportunity to make laboratory-based measurements of propulsion accessible to clinicians.

An initial step in this endeavor was to determine the most significant metrics of propulsion as it relates to distance walked during a 6MWT, a popular outcome measure linked to long distance walking function and increased quality of life.
Thus, the first aim of this dissertation was to evaluate the effects of propulsion metrics on long distance walking function using statistical learning methods.
The results showed that braking magnitude and impulse of both the paretic and nonparetic limbs were significant predictors of total distance walked, differing from the common focus on propulsion.

A following step was to confirm that these measurements can be accessible to clinicians in a cost-effective manner.
The second aim of this dissertation was to validate the accuracy of an IMU based machine learning algorithm in estimating propulsion metrics versus laboratory-based equipment.
The results showed that both propulsion metrics and entire APGRF curves cannot accurately be estimated with the method used.

While the insight of braking metrics as a predictor of long distance walking function is useful, more work can be done to tailor accessible technologies to the needs of clinicians.
