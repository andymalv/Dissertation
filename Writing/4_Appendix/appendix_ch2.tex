\chapter{Code for Chapter 2}
\label{appendix}
\thispagestyle{myheadings}

\lstset{
	xleftmargin=0.05\textwidth
}

This appendix contains the programming code for Chapter 2: Evaluation of propulsion metrics on long distance walking function.
All code was written in MATLAB 2019a (Mathworks).

\section{T1\_ExtractData.m}
Code to extract motion capture data:

\begin{lstlisting}[language=Matlab]
clear
close all
clc

%%

cd 'filepath'

% Get ID and paretic side information
%%% this was done manually and put in a .csv file; just reading that
df = table2struct(readtable('Subject_Info.csv'));

All.ID(:,1) = string({df.ID});
All.Paretic(:,1) = string({df.PareticSide});
All.Mass(:,1) = [df.Weight];

Subject = All.ID;
Side = ["Paretic", "Nonparetic"];
Day = ["NS", "S"];
Condition = ["Pre", "Post"];

%% Read in Ascii files

%%% reads in each file from ID list

for i = 1:length(Subject)
    
    sub = Subject(i);
    folder.Data = 'filepath';
    
    % Create file names
    file1 = strcat(sub, '_Left.txt');
    file2 = strcat(sub, '_Right.txt');
    
    % Set up file paths
    folder.NS.Pre = strcat(folder.Data, sub, '\No Suit\4 Exported Data\Pre');
    folder.NS.Post = strcat(folder.Data, sub, '\No Suit\4 Exported Data\Post');
    
    folder.S.Pre = strcat(folder.Data, sub, '\Suit\4 Exported Data\Pre');
    folder.S.Post = strcat(folder.Data, sub, '\Suit\4 Exported Data\Post');
    
    % Check which side is paretic
    paretic = All.Paretic(i);
    
    for j = 1:length(Day)

        for k = 1:length(Condition)
            
            day = Day(j);
            cond = Condition(k);
            
            cd(folder.(day).(cond));
            
            % Import files
            left = importdata(file1);
            right = importdata(file2);
            
            if paretic == 'Right'
                
                All.(sub).(day).(cond).Paretic = right;
                All.(sub).(day).(cond).Nonparetic = left;
                
            else
                
                All.(sub).(day).(cond).Paretic = left;
                All.(sub).(day).(cond).Nonparetic = right;
                
            end
            

        end
     
    end

    clear file1 file2 left right sub sns prpo folder paretic
    
end


cd 'filepath'


%% Extract and Normalize GRFs

% g = 9.8;

for i = 1:length(Subject) % subject
    
    sub = Subject(i);
    
    for j = 1:length(Day) % day
        
        day = Day(j);
        
        for k = 1:length(Condition) % condition
            
            con = Condition(k);
            
            for l = 1:length(Side) % side
                
                side = Side(l);
                
                % Get data and replace NaNs w/ zeros
                blah = All.(sub).(day).(con).(side).data(:,2:end);
                blah(isnan(blah)) = 0;
                
                % Initialize counter variable
                m = 1;
                
                % For each stride (set of x, y and z)
                for n = 1:3:size(blah,2)
                    
                    % Get single stride
                    stride = blah(:,n:(n+2)); 
                    
                    % Remove zeros from arrays
                    x = nonzeros(stride(:,1));
                    y = nonzeros(stride(:,2));
                    z = nonzeros(stride(:,3));
                    
                    % Resample to 100 points
                    X = resample(x,100,length(x));
                    Y = resample(y,100,length(y));
                    Z = resample(z,100,length(z));
                    
                    % Move to structure
                    All.(sub).(day).(con).(side).GRF.X(:,m) = X*100;
                    All.(sub).(day).(con).(side).GRF.Y(:,m)  = Y*100;
                    All.(sub).(day).(con).(side).GRF.Z(:,m)  = Z*100;
                    
                    % Add to counter
                    m = m+1;
                    
                end
                
                clear stride m x y z X Y Z
                
            end
            
            clear side blah
            
        end
        
        clear con
        
    end
    
    clear day
    
end

clear sub g


%%
save('All.mat', 'All')

\end{lstlisting}
\pagebrake

\section{T2\_CalculateMetrics.m}
Code to calculate metrics of APGRF curve:

\begin{lstlisting}[language=Matlab]
clear
close all
clc

% Load file w/ data
load('All.mat')

% Get number of subjects
Subject = All.ID;
Side = ["Paretic", "Nonparetic"];
Day = ["NS", "S"];
Condition = ["Pre", "Post"];


%% Calculate metrics - % stance time

% For each subject
for i = 1:length(Subject)
    
    sub = Subject(i);
    
    % For each day
    for j = 1:length(Day)
        
        day = Day(j);
        
        % For each condition
        for k = 1:length(Condition)
            
            con = Condition(k);
            
            % For each side
            for l = 1:length(Side)
                
                side = Side(l);
                
                % Get APGRF
                obj = All.(sub).(day).(con).(side).GRF.Y;
                
                % For each stride
                for n = 1:size(obj,2)
                    
                    % Get stride
                    stride = obj(:,n);
                    
                    %% Find when max braking and propulsion magnitude occurs for each stride
                    prop.peak.mag(n,:) = max(stride(40:end));  % assume prop happens after 40% stance
                    brake.peak.mag(n,:) = min(stride(10:50)); % assume braking happens 10-50% stance
                    
                    brake.peak.time(n,:) = find(stride == brake.peak.mag(n)); % location of peak brake per stride
                    prop.peak.time(n,:) = find(stride == prop.peak.mag(n)); % location of peak prop per stride
                    
                    %% Find propulsion onset (zero crossing)
                    section = stride(brake.peak.time(n):prop.peak.time(n)); % get array from braking to prop maxes
                    pos = find(section >= 0, 1); % find first propulsion onset value
                    
                    % If there's no braking phase
                    if brake.peak.mag(n) >= 0
                        
                        prop.onset(n,:) = brake.peak.time(n);
                        
                    % If there's a point at y=0 (highly unlikely)
                    elseif section(pos) == 0
                        
                        prop.onset(n,:) = brake.peak.time(n) + pos - 1;
                        
                    % If there's no propulsion
                    elseif isempty(pos)
                        
                        prop.onset(n,:) = 0;
                        
                    % If pos ~= 0 (very likely)
                    else
                        
                        neg = pos - 1; % get last negative value
                        rise = section(pos) + (-section(neg)); % calculate change in y
                        run = pos - neg; % calculate change in x (probably 1)
                        slope = rise / run; % find slope b/w last negative and first positive points
                        chy = abs(section(neg)); % change in y from last negative to y=0
                        prchy = chy / slope; % percent change in y compared to slope
                        chx = run * prchy; % percent change in x from last negative to y=0
                        cross = neg + chx; % change in x from last negative to y=0
                        prop.onset(n,:) = brake.peak.time(n) + cross - 1; % calculate x when y=0
                        
                    end
                    
                    % Clear variables for future use
                    clear section pos neg rise run slope chy prchy chx cross
                    
                    %% Find end of propulsion
                    section = stride(prop.peak.time(n):end); % from peak propulsion to end of stride
                    neg = find(section <=0,1); % find that dip below zero at the end of stance
                    
                    % If there's no dip below zero at the end
                    if isempty(neg)
                        
                        prop.end(n,:) = length(stride);
                        
                    % If there's a point at y=0 (highly unlikely)
                    elseif section(neg) == 0
                        
                        prop.end(n,:) = prop.peak.time(n) + neg - 1;
                        
                    % If there's no propulsion
                    elseif prop.onset(n,:) == 0
                        
                        prop.end(n,:) = 0;
                        
                    % If neg ~= 0 (very likely)
                    else
                        
                        pos = neg - 1; % get last positive value
                        rise = section(pos) + (-section(neg)); % calculate change in y
                        run = neg - pos; % calculate change in x (probably 1)
                        slope = rise / run; % find neg b/w last negative and first positive points
                        chy = rise - (-section(neg)); % change in y from last negative to y=0
                        prchy = chy / slope; % percent change in y compared to slope
                        chx = run * prchy; % percent change in x from last negative to y=0
                        cross = pos + chx; % change in x from last negative to y=0
                        prop.end(n,:) = prop.peak.time(n) + cross - 1; % calculate x when y=0
                        
                    end
                    
                    % Clear variables for future use
                    clear section pos neg rise run slope chy prchy chx cross
                    
                    %% Calculate total propulsion time
                    prop.time(n,:) = prop.end(n) - prop.onset(n);
                    
                    %% Calculate propulsion impulse
                    
                    % If there's no proulsion
                    if prop.onset(n,:) == 0
                        
                        prop.impulse(n,:) = 0;
                        
                    % If there is propulsion
                    else
                        
                        section = stride(ceil(prop.onset(n)):ceil(prop.end(n))); % get propulsion part
                        prop.impulse(n,:) = trapz(section) / length(stride); % take integral, divide by total number of points
                        
                    end
                    
                    % Clear variables for future use
                    clear section
                    
                    %% Find braking onset
                    section = stride(1:brake.peak.time(n)); % get array from beginning to peak braking
                    pos = find(section >=0, 1, 'last'); % find all positive values - last one will be right before braking
                    
                    % If there's no braking phase
                    if brake.peak.mag(n) >= 0
                        
                        brake.onset(n,:) = NaN;
                        
                     % If there's no propulsion before peak braking
                    elseif isempty(pos)
                        
                        brake.onset(n,:) = 1;  % set braking onset at beginning
                        
                    % If there's some propulsion before peak braking
                    else
                        
                        % If there's a point at y=0 (highly unlikely)
                        if pos == 0
                            
                            brake.onset(n,:) = pos - 1;
                            
                            % If pos ~= 0 (very likely)
                        else
                            
                            neg = pos + 1; % get first negative value
                            rise = section(pos) + (-section(neg)); % calculate change in y
                            run = pos - neg; % calculate change in x (probably -1)
                            slope = rise / run; % find slope b/w last positive and first negative points
                            chy = section(pos); % change in y from last positive to y=0
                            prchy = chy / slope; % percent change in y compared to slope
                            chx = run * prchy; % percent change in x from last positive to y=0
                            cross = pos + chx; % change in x from last negative to y=0
                            brake.onset(n,:) = cross; % calculate x when y=0
                            
                        end
                        
                    end
                    
                    % Clear variables for future use
                    clear section pos neg rise run slope chy prchy chx cross
                    
                    %% Calculate total braking time
                    
                    % If there's no braking phase
                    if brake.peak.mag(n) >= 0
                        
                        brake.time(n,:) = 0;
                        
                    % If there IS a braking phase
                    else
                        
                        brake.time(n,:) = prop.onset(n) - brake.onset(n);
                        
                    end
                    
                    %% Calculate braking impulse
                    
                    % If there's no braking phase
                    if brake.peak.mag(n) >= 0
                        
                        brake.impulse(n,:) = 0;
                        
                    % If there IS a braking phase
                    else
                        
                        section = stride(ceil(brake.onset(n)):floor(prop.onset(n))); % get braking part
                        brake.impulse(n,:) = trapz(section) / length(stride); % take integral, divide by total number of points
                        
                    end
                    
                    % Clear variables for future use
                    clear section
                    
                    %% Calculate time b/w peak braking and peak propulsion
                    time_bw(n,:) = prop.peak.time(n) - brake.peak.time(n);
                    
                end
                
                %% Transfer things to the structure
                
                % Braking onset time
                All.(sub).(day).(con).(side).GRF.metrics.brake.onset.raw...
                    = brake.onset; % raw data
                All.(sub).(day).(con).(side).GRF.metrics.brake.onset.mean...
                    = mean(brake.onset); % mean
                All.(sub).(day).(con).(side).GRF.metrics.brake.onset.median...
                    = median(brake.onset); % median
                All.(sub).(day).(con).(side).GRF.metrics.brake.onset.std...
                    = std(brake.onset); % standard deviation
                
                % Braking peak time
                All.(sub).(day).(con).(side).GRF.metrics.brake.peak_time.raw...
                    = brake.peak.time; % raw data
                All.(sub).(day).(con).(side).GRF.metrics.brake.peak_time.mean...
                    = mean(brake.peak.time); % mean
                All.(sub).(day).(con).(side).GRF.metrics.brake.peak_time.median...
                    = median(brake.peak.time); % median
                All.(sub).(day).(con).(side).GRF.metrics.brake.peak_time.std...
                    = std(brake.peak.time); % standard deviation
                
                % Braking peak magnitude
                All.(sub).(day).(con).(side).GRF.metrics.brake.peak_mag.raw...
                    = brake.peak.mag; % raw data
                All.(sub).(day).(con).(side).GRF.metrics.brake.peak_mag.mean...
                    = mean(brake.peak.mag); % mean
                All.(sub).(day).(con).(side).GRF.metrics.brake.peak_mag.median...
                    = median(brake.peak.mag); % median
                All.(sub).(day).(con).(side).GRF.metrics.brake.peak_mag.std...
                    = std(brake.peak.mag); % standard deviation
                
                % Braking impulse
                All.(sub).(day).(con).(side).GRF.metrics.brake.impulse.raw...
                    = brake.impulse; % raw data
                All.(sub).(day).(con).(side).GRF.metrics.brake.impulse.mean...
                    = mean(brake.impulse); % mean
                All.(sub).(day).(con).(side).GRF.metrics.brake.impulse.median...
                    = median(brake.impulse); % median
                All.(sub).(day).(con).(side).GRF.metrics.brake.impulse.std...
                    = std(brake.impulse); % standard deviation
                
                % Braking total time
                All.(sub).(day).(con).(side).GRF.metrics.brake.total_time.raw...
                    = brake.time; % raw data
                All.(sub).(day).(con).(side).GRF.metrics.brake.total_time.mean...
                    = mean(brake.time); % mean
                All.(sub).(day).(con).(side).GRF.metrics.brake.total_time.median...
                    = median(brake.time); % median
                All.(sub).(day).(con).(side).GRF.metrics.brake.total_time.std...
                    = std(brake.time); % standard deviation
                
                % Propulsion onset time
                All.(sub).(day).(con).(side).GRF.metrics.prop.onset.raw...
                    = prop.onset; % raw data
                All.(sub).(day).(con).(side).GRF.metrics.prop.onset.mean...
                    = mean(prop.onset); % mean
                All.(sub).(day).(con).(side).GRF.metrics.prop.onset.median...
                    = median(prop.onset); % median
                All.(sub).(day).(con).(side).GRF.metrics.prop.onset.std...
                    = std(prop.onset); % standard deviation
                
                % Propulsion peak time
                All.(sub).(day).(con).(side).GRF.metrics.prop.peak_time.raw...
                    = prop.peak.time; % raw data
                All.(sub).(day).(con).(side).GRF.metrics.prop.peak_time.mean...
                    = mean(prop.peak.time); % mean
                All.(sub).(day).(con).(side).GRF.metrics.prop.peak_time.median...
                    = median(prop.peak.time); % median
                All.(sub).(day).(con).(side).GRF.metrics.prop.peak_time.std...
                    = std(prop.peak.time); % standard deviation
                
                % Propulsion peak magnitude
                All.(sub).(day).(con).(side).GRF.metrics.prop.peak_mag.raw...
                    = prop.peak.mag; % raw data
                All.(sub).(day).(con).(side).GRF.metrics.prop.peak_mag.mean...
                    = mean(prop.peak.mag); % mean
                All.(sub).(day).(con).(side).GRF.metrics.prop.peak_mag.median...
                    = median(prop.peak.mag); % median
                All.(sub).(day).(con).(side).GRF.metrics.prop.peak_mag.std...
                    = std(prop.peak.mag); % standard deviation
                
                % Propulsion impulse
                All.(sub).(day).(con).(side).GRF.metrics.prop.impulse.raw...
                    = prop.impulse; % raw data
                All.(sub).(day).(con).(side).GRF.metrics.prop.impulse.mean...
                    = mean(prop.impulse); % mean
                All.(sub).(day).(con).(side).GRF.metrics.prop.impulse.median...
                    = median(prop.impulse); % median
                All.(sub).(day).(con).(side).GRF.metrics.prop.impulse.std...
                    = std(prop.impulse); % standard deviation
                
                % Propulsion total time
                All.(sub).(day).(con).(side).GRF.metrics.prop.total_time.raw...
                    = prop.time; % raw data
                All.(sub).(day).(con).(side).GRF.metrics.prop.total_time.mean...
                    = mean(prop.time); % mean
                All.(sub).(day).(con).(side).GRF.metrics.prop.total_time.median...
                    = median(prop.time); % median
                All.(sub).(day).(con).(side).GRF.metrics.prop.total_time.std...
                    = std(prop.time); % standard deviation
                
                % Time b/w peaks
                All.(sub).(day).(con).(side).GRF.metrics.bw.time.raw...
                    = time_bw; % raw data
                All.(sub).(day).(con).(side).GRF.metrics.bw.time.mean...
                    = mean(time_bw); % mean
                All.(sub).(day).(con).(side).GRF.metrics.bw.time.median...
                    = median(time_bw); % median
                All.(sub).(day).(con).(side).GRF.metrics.bw.time.std...
                    = std(time_bw); % standard deviation
                
                
                %% Clear all variables
                
                clear obj prop brake time_bw section pos neg rise run slope chy...
                    prchy chx cross
                
            end
             
        end
        
    end
    
end




%% Save
save('All.mat', 'All')


\end{lstlisting}
\pagebreak

















